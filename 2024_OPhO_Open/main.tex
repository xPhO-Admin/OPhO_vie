\documentclass[11pt]{article}
\usepackage{physoly}
\usepackage[utf8]{vietnam} % For Vietnamese translation
\usepackage{multicol}
\usepackage{wasysym}
\usepackage{titling}
\usepackage{epstopdf}
\usepackage{physics}
\usepackage{enumitem}
\usepackage{lipsum}
% \usepackage[T1]{fontenc}
\usepackage{siunitx}
\usepackage{dcolumn, amsmath, amssymb}
\usepackage[section]{placeins}
\usepackage{hyperref}
\setlength{\parskip}{\baselineskip}%
\usepackage[inline]{asymptote}
\usepackage{multirow}
\setlength{\parindent}{0pt}
\renewcommand{\quote}{\list{}{\rightmargin=\leftmargin\topsep=0pt}\item\relax}
\usepackage{changepage}   % for the adjustwidth environment
\usetikzlibrary{positioning,fadings,through}

\definecolor{left} {HTML}{001528}

\usepackage{hyperref}
\hypersetup{
    colorlinks=true,
    linkcolor=blue,
    filecolor=magenta,      
    urlcolor=blue,
}
\usepackage{tikz}
\tikzset{%
  3d/.unknown/.code={%
    \ifx\cx\relax%
      \let\cx=\pgfkeyscurrentname%
    \else%
       \ifx\cy\relax%
         \let\cy=\pgfkeyscurrentname%
       \else%
         \let\cz=\pgfkeyscurrentname%
       \fi%
     \fi% 
    }
}
\tikzdeclarecoordinatesystem{3d}{%
  \let\cx=\relax\let\cy=\relax\let\cz=\relax%
  \tikzset{3d/.cd,#1}%
  % Not a proper perspective calculation
  \pgfmathparse{1.05^(\cx)*1.05^(\cz)}\let\cf=\pgfmathresult%
  \pgfpointxyz{(\cx)*\cf}{(\cy)*\cf}{(\cz)*\cf}%
}
\usetikzlibrary{shapes.geometric, arrows}
\tikzstyle{arrow} = [thick,->,>=stealth]
\usetikzlibrary{decorations.pathmorphing,patterns}
\tikzstyle{spring}=[thick,decorate,decoration={zigzag,pre length=0.1cm,post
  length=0.1cm,segment length=6}]
\usepackage{subfigure}

\usepackage{tikz}
\def\width{19}
\def\hauteur{15}
\usetikzlibrary{shapes,positioning,intersections,quotes}
\usepackage{fancyhdr}

\pagestyle{fancy}
\fancyhf{}
\rhead{Online Physics Olympiad 2024 - Đáp án vòng Open}
\lhead{Ngày 12 - 14, tháng 8, năm 2024}
\rfoot{Trang \thepage}
\title{Đề OPhO}

\begin{document}

\begin{titlepage}
    \begin{center}
        
        \Huge
        \textbf{2024 Online Physics Olympiad: \\
        Đáp án vòng Open}
        
        \vspace{0.5cm}
        
        \begin{center}
            \includegraphics[width=0.6\linewidth]{OPhO Logo, White BKGD.png}
        \end{center}
        \end{center}


\section*{Nhà tài trợ}

\vspace{-0.5cm}

Cuộc thi này sẽ không thể thành công nếu thiếu đi sự trợ giúp từ các nhà tài trợ, những người đã đóng góp rất nhiều cho vật lý, toán và giáo dục.
% This competition could not be possible without the help of our sponsors, who are all doing great things in physics, math, and education.
\begin{center}
    \includegraphics[width = 4cm]{sponsors/jane-street.png}\qquad\includegraphics[width = 5cm]{sponsors/wolfram.png}\qquad\includegraphics[width = 5cm]{sponsors/awesome-math.png}\\ 
    
    \includegraphics[width=2cm]{sponsors/btbb.png}\qquad\includegraphics[width=2.5cm]{sponsors/physolymp.jpg}\qquad\includegraphics[width=4cm]{sponsors/aops.png}
\end{center}
\end{titlepage}
\newpage
\section*{Hướng dẫn}
Nếu bạn muốn yêu cầu làm rõ, vui lòng sử dụng \href{https://forms.gle/adb6hNCrQFhHVvVe6}{biểu mẫu này}. Để xem tất cả các giải thích, xem \href{https://docs.google.com/document/d/1W71dFM-XDmuTxXJ_MJ3QfSdE_xxISvIw58FZh1wYEIk/edit}{tài liệu này}.
\begin{itemize}
    \item Sử dụng $g=9.8\;\mathrm{m/s^2}$ trong cuộc thi này, \textbf{trừ khi có quy định khác}. Xem bảng hằng số ở trang tiếp theo để biết các hằng số khác.
    \item Bài kiểm tra này có 35 câu hỏi ngắn. Mỗi bài toán sẽ có ba lần thử.
    \item Trọng số của mỗi câu hỏi phụ thuộc vào hệ thống chấm điểm của chúng tôi được tìm thấy \href{https://opho.physoly.tech/static/files/rules24.pdf}{ở đây}. Nói đơn giản, các câu hỏi sau có giá trị cao hơn, và tổng số điểm từ một câu hỏi nhất định sẽ giảm theo số lần bạn thử giải bài toán cũng như số đội giải được nó.
    \item Bất kỳ thành viên nào trong đội đều có thể nộp bài. Việc chia nhỏ công việc hay làm mỗi bài toán cùng nhau là tùy thuộc vào bạn. Lưu ý rằng sau khi bạn đã nộp bài, các đồng đội của bạn phải làm mới trang của họ trước khi họ có thể thấy nó.
    \item Câu trả lời nên chứa \textbf{ba} chữ số có nghĩa, trừ khi có quy định khác. Tất cả các câu trả lời trong phạm vi \(1\%\) sẽ được chấp nhận.
    \item Khi nộp câu trả lời bằng ký hiệu khoa học, vui lòng sử dụng dạng số mũ. Nói cách khác, nếu câu trả lời của bạn cho một bài toán là $A\times 10^B$, vui lòng nhập $A\text{e}B$ vào cổng nộp bài.
    \item Một máy tính cầm tay khoa học hoặc đồ thị tiêu chuẩn \textit{có thể} được sử dụng. Các hệ thống đại số máy tính và công nghệ như Wolfram Alpha hoặc TI Nspire sẽ không cần thiết, nhưng có thể được sử dụng.
    \item Bạn \textit{được phép} sử dụng Wikipedia hoặc sách trong kỳ thi này. Hỏi xin trợ giúp trên các diễn đàn trực tuyến hoặc từ giáo viên của bạn sẽ bị coi là gian lận và có thể dẫn đến việc cấm tham gia các cuộc thi trong tương lai.
    \item Những người đạt điểm cao nhất từ cuộc thi này sẽ đủ điều kiện tham gia \textit{Cuộc thi Mời của Olympic Vật lý Trực tuyến}, là một kỳ thi theo phong cách olympic. Thông tin thêm sẽ được cung cấp cho những người đủ điều kiện sau khi kết thúc \textit{Cuộc thi Mở}.
    \item Nói chung, trả lời bằng đơn vị SI (mét, giây, kilogram, watt, v.v.) trừ khi có quy định khác. Vui lòng nhập tất cả các góc bằng độ trừ khi có quy định khác.
    \item Nếu câu hỏi yêu cầu đưa ra câu trả lời dưới dạng phần trăm và câu trả lời của bạn là ``$x$\%”, vui lòng nhập giá trị $x$ vào biểu mẫu nộp bài.
    \item Không đặt đơn vị trong câu trả lời của bạn trên cổng nộp bài! Nếu câu trả lời của bạn là ``$x$ mét”, chỉ nhập giá trị $x$ vào cổng nộp bài.
    \item \textbf{Không truyền đạt thông tin cho bất kỳ ai khác ngoài các thành viên trong đội của bạn trước ngày 25 tháng 8 năm 2024}.
\end{itemize}

\newpage
\section*{Danh mục hằng số}
\begin{multicols}{2}
\begin{itemize}
    \item Khối lượng proton, $m_p = 1.67\cdot 10^{-27}\;\mathrm{kg}$
    \item Khối lượng neutron, $m_n = 1.67 \cdot 10^{-27}\;\mathrm{kg}$
    \item Khối lượng electron, $m_e = 9.11 \cdot 10^{-31}\;\mathrm{kg}$
    \item Hằng số Avogadro, $N_0 = 6.02 \cdot 10^{23}\;\mathrm{mol^{-1}}$
    \item Hằng số khí lý tưởng, $R = 8.31\;\mathrm{J/(mol\cdot K)}$
    \item Hằng số Boltzmann, $k_B = 1.38\cdot 10^{-23}\;\mathrm{J/K}$
    \item Điện tích hạt electron, $e = 1.60 \cdot 10^{-19}\;\mathrm{C}$
    \item 1 electron volt, $1\;\mathrm{eV} = 1.60\cdot 10^{-19}\;\mathrm{J}$
    \item Vận tốc ánh sáng, $c = 3.00 \cdot 10^8\;\mathrm{m/s}$
    \item Hằng số hấp dẫn, \[G = 6.67\cdot 10^{-11}\;\mathrm{(N\cdot m^2)/kg^2}\]
    \item Khối lượng mặt trời
    \[M_{\odot} = 1.988\cdot 10^{30}\;\mathrm{kg}\]
    \item Gia tốc trọng trường, $g = 9.8\;\mathrm{m/s^2}$
    \item 1 đơn vị nguyên tử khối, \[1\;\mathrm{u} = 1.66 \cdot 10^{-27}\;\mathrm{kg} = 931\;\mathrm{MeV/c^2}\]
    \item Hằng số Planck, 
        \[h = 6.63 \cdot 10^{-34}\;\mathrm{J\cdot s} = 4.41\cdot 10^{-15}\;\mathrm{eV\cdot s}\]
    \item Hằng số điện môi của chân không, \[\epsilon_0 = 8.85 \cdot 10^{-12}\;\mathrm{C^2/(N\cdot m^2)}\]
    \item Hằng số lực Coulomb,
    \[k = \frac{1}{4\pi\epsilon_0} = 8.99 \cdot 10^9\;\mathrm{(N\cdot m^2)/C^2}\]
    \item Độ từ thẩm của chân không, \[\mu_0 = 4\pi\cdot 10^{-7}\;\mathrm{T\cdot m/A}\]
    \item Hằng số từ, 
    \[ \frac{\mu_0}{4\pi} = 1\cdot 10^{-7}\;\mathrm{(T\cdot m)/A}\]
    \item Áp suất 1 atmosphere ,
    \[1\;\mathrm{atm} = 1.01 \cdot 10^5\;\mathrm{N/m^2} = 1.01\cdot 10^5\;\mathrm{Pa}\]
    \item Hằng số dịch chuyển Wien, $b = 2.9\cdot 10^{-3}\;\mathrm{m\cdot K}$
    \item Hằng số Stefan-Boltzmann, \[\sigma = 5.67\cdot 10^{-8}\;\mathrm{W/m^2/K^4}\]
\end{itemize}
\end{multicols}
\newpage
\normalsize
\begin{center}
    \Large 
    \textbf{Bài tập}
\end{center}

\begin{solution}
    The car's initial velocity is $v_0 =X/T$. Using $v^2 = 2 a \Delta x$, we find its acceleration is $a=\frac{X}{2T^2}$. Therefore, the distance the car travels in time $T$ is 

    $$v_0 T - \frac{1}{2} a T^2 = \frac{3}{4} X= 30.0 \ m$$

    You do not get hit by the car, and you are $\boxed{2.0 \ \mathrm{m}}$ away from the car when you finish crossing the street.

\end{solution}
\begin{solution}
    The car's initial velocity is $v_0 =X/T$. Using $v^2 = 2 a \Delta x$, we find its acceleration is $a=\frac{X}{2T^2}$. Therefore, the distance the car travels in time $T$ is 

    $$v_0 T - \frac{1}{2} a T^2 = \frac{3}{4} X= 30.0 \ m$$

    You do not get hit by the car, and you are $\boxed{2.0 \ \mathrm{m}}$ away from the car when you finish crossing the street.

\end{solution}

\begin{problem}
{\textbf{\textsc{Ambidexterity}}}
When plane-polarized light $\lambda = 500\;\mathrm{nm}$ is passed through a solution with chiral molecules (eg. glucose/DNA), the exiting light is observed to have been rotated by an angle $\Delta \theta$. In chemistry, this optical rotation is measured with a polarimeter device. It is used to measure the relative abundances of left-handed and right-handed molecules, each having their own refractive index $n_L = 1.333333$ and $n_R = 1.333338$ affecting left and right circularly polarized respectively. 

\begin{center}
\includegraphics[width=.8\textwidth]{problems/figures/opticalRotationSchematic.png}
\end{center}

If the length of the solution container is $L = 0.15\;\mathrm{m}$, by which angle will the polarization of light rotate, i.e. what is the \textbf{total} optical rotation $\Delta \theta$, in radians? Input the smallest positive answer to this question.
\end{problem}
\begin{problem}
{\textbf{\textsc{Ambidexterity}}}
When plane-polarized light $\lambda = 500\;\mathrm{nm}$ is passed through a solution with chiral molecules (eg. glucose/DNA), the exiting light is observed to have been rotated by an angle $\Delta \theta$. In chemistry, this optical rotation is measured with a polarimeter device. It is used to measure the relative abundances of left-handed and right-handed molecules, each having their own refractive index $n_L = 1.333333$ and $n_R = 1.333338$ affecting left and right circularly polarized respectively. 

\begin{center}
\includegraphics[width=.8\textwidth]{problems/figures/opticalRotationSchematic.png}
\end{center}

If the length of the solution container is $L = 0.15\;\mathrm{m}$, by which angle will the polarization of light rotate, i.e. what is the \textbf{total} optical rotation $\Delta \theta$, in radians? Input the smallest positive answer to this question.
\end{problem}

\begin{problem}
{\textbf{\textsc{Ball Drop}}} A ball with uniform density $\rho_b$ is placed on the surface of a pool with depth $d$ and liquid density $\rho_p < \rho_b$. Another identical ball is lifted a height $h$ above the pool, and then both balls are released at the same time. In order for both balls to touch the bottom of the pool at the same time, the condition $d = nh$ must be met for some dimensionless $n$ that depends on the values of $\rho_p$ and $\rho_b$. If we define
$$r = \frac{\rho_b - \rho_p}{\rho_b}$$
Then we can express $n$ as
$$n = \frac{Ar^3 + Br^2 + Cr}{Dr^2+Er+F}$$
Where $A, B, C, D, E, F$ are all nonzero integers, $\gcd(A,B,C,D,E,F) = 1$, and $A>0$. What is $A + B + C + D + E + F$?
You may assume that the only forces present are gravity and the buoyant force from the pool. The airborne ball retains all of its energy as it enters the pool.
% \FloatBarrier
% \begin{figure}[!htbp]
%     \centering
%     \includegraphics[scale=0.42]{problems/figures/ballDiagram.png}
% \end{figure}
% \FloatBarrier
\end{problem}
\begin{problem}
{\textbf{\textsc{Ball Drop}}} A ball with uniform density $\rho_b$ is placed on the surface of a pool with depth $d$ and liquid density $\rho_p < \rho_b$. Another identical ball is lifted a height $h$ above the pool, and then both balls are released at the same time. In order for both balls to touch the bottom of the pool at the same time, the condition $d = nh$ must be met for some dimensionless $n$ that depends on the values of $\rho_p$ and $\rho_b$. If we define
$$r = \frac{\rho_b - \rho_p}{\rho_b}$$
Then we can express $n$ as
$$n = \frac{Ar^3 + Br^2 + Cr}{Dr^2+Er+F}$$
Where $A, B, C, D, E, F$ are all nonzero integers, $\gcd(A,B,C,D,E,F) = 1$, and $A>0$. What is $A + B + C + D + E + F$?
You may assume that the only forces present are gravity and the buoyant force from the pool. The airborne ball retains all of its energy as it enters the pool.
% \FloatBarrier
% \begin{figure}[!htbp]
%     \centering
%     \includegraphics[scale=0.42]{problems/figures/ballDiagram.png}
% \end{figure}
% \FloatBarrier
\end{problem}

\begin{solution}
The first key idea here is that the magnetic flux through the solenoid $\Phi = LI$ is a conserved quantity, since $\varepsilon = \frac{d \Phi}{dt}$ is necessarily zero to stop current from blowing up. The second key idea is that this is essentially just a conservation-of-energy question, with the decrease in magnetic field energy during the transit of the core through the solenoid being converted to kinetic energy. As a result, the optimal time to shut down the solenoid occurs when magnetic field energy is lowest i.e. when the core is completely inside the solenoid.
\newline
\newline
Since B is conserved and $\mu \gg \mu_0$, there is essentially no magnetic field energy when the solenoid should be shut down.
As a result, we have
$$E_B = \frac{\bigg (\frac{\mu_0 N I}{L}\bigg )^2}{2 \mu_0}Al = {\frac{1}{2}}mv^2$$
Plugging in yields $\boxed{v = 179.4 \;\mathrm{m/s}}$
\end{solution}
\begin{solution}
The first key idea here is that the magnetic flux through the solenoid $\Phi = LI$ is a conserved quantity, since $\varepsilon = \frac{d \Phi}{dt}$ is necessarily zero to stop current from blowing up. The second key idea is that this is essentially just a conservation-of-energy question, with the decrease in magnetic field energy during the transit of the core through the solenoid being converted to kinetic energy. As a result, the optimal time to shut down the solenoid occurs when magnetic field energy is lowest i.e. when the core is completely inside the solenoid.
\newline
\newline
Since B is conserved and $\mu \gg \mu_0$, there is essentially no magnetic field energy when the solenoid should be shut down.
As a result, we have
$$E_B = \frac{\bigg (\frac{\mu_0 N I}{L}\bigg )^2}{2 \mu_0}Al = {\frac{1}{2}}mv^2$$
Plugging in yields $\boxed{v = 179.4 \;\mathrm{m/s}}$
\end{solution}

\begin{problem}{\textbf{\textsc{Into Orbit}}} A cannon is fixed to the top of a platform of height $R = 6.00 \times 10^6\;\mathrm{m}$, which sits on a planet of mass $M = 6.00 \times 10^{24}\;\mathrm{kg}$ and radius $R$. Both the cannon and the platform it rests on have negligible mass, and the cannon's chamber is angled horizontally relative to the planet below it. The cannon then fires a cannonball of with mass $m = 45\;\mathrm{kg}$ through a chamber of length $l=3\;\mathrm{m}$ at a constant acceleration such that the cannonball is able to successfully enter an elliptical orbit around the planet. What is the minimum force that must be applied by the cannon on the cannonball in order to make this possible? You may assume that both the planet and the platform do not move during this process.\end{problem}
\begin{problem}{\textbf{\textsc{Into Orbit}}} A cannon is fixed to the top of a platform of height $R = 6.00 \times 10^6\;\mathrm{m}$, which sits on a planet of mass $M = 6.00 \times 10^{24}\;\mathrm{kg}$ and radius $R$. Both the cannon and the platform it rests on have negligible mass, and the cannon's chamber is angled horizontally relative to the planet below it. The cannon then fires a cannonball of with mass $m = 45\;\mathrm{kg}$ through a chamber of length $l=3\;\mathrm{m}$ at a constant acceleration such that the cannonball is able to successfully enter an elliptical orbit around the planet. What is the minimum force that must be applied by the cannon on the cannonball in order to make this possible? You may assume that both the planet and the platform do not move during this process.\end{problem}

\begin{problem}
{\textbf{\textsc{Roundabout 1}}} Một dốc có chiều dài $d$ được nâng lên một góc $\theta$ ($0^{\circ} < \theta < 90^{\circ}$) so với mặt phẳng ngang. Một khối có khối lượng $m$ được đặt ở đỉnh của dốc, với hệ số ma sát giữa khối và dốc là $\mu$. Khi khối di chuyển đến đáy của dốc, nó giữ nguyên vận tốc khi được chuyển tiếp mượt mà lên một đường tròn không ma sát với bán kính $d$ và góc nghiêng $\theta$, quay trên đường tròn mà không bị trượt ra ngoài. Một \emph{solution} là một tập hợp các giá trị $\{d, \mu, \theta\}$ dẫn đến tình huống đã mô tả ở trên. Giá trị lớn nhất của $\theta$ (tính bằng độ) mà vẫn tồn tại giải pháp là gì? 

\end{problem}
\begin{problem}
{\textbf{\textsc{Roundabout 1}}} Một dốc có chiều dài $d$ được nâng lên một góc $\theta$ ($0^{\circ} < \theta < 90^{\circ}$) so với mặt phẳng ngang. Một khối có khối lượng $m$ được đặt ở đỉnh của dốc, với hệ số ma sát giữa khối và dốc là $\mu$. Khi khối di chuyển đến đáy của dốc, nó giữ nguyên vận tốc khi được chuyển tiếp mượt mà lên một đường tròn không ma sát với bán kính $d$ và góc nghiêng $\theta$, quay trên đường tròn mà không bị trượt ra ngoài. Một \emph{solution} là một tập hợp các giá trị $\{d, \mu, \theta\}$ dẫn đến tình huống đã mô tả ở trên. Giá trị lớn nhất của $\theta$ (tính bằng độ) mà vẫn tồn tại giải pháp là gì? 

\end{problem}

\begin{problem}
{\textbf{\textsc{Roundabout 2}}} Khi khối đi quanh đường tròn, nó được đẩy nhẹ theo phương vuông góc với vận tốc hiện tại của nó và song song với mặt phẳng của đường tròn, gây ra hiện tượng dao động với chu kỳ $T$. Giá trị nhỏ nhất có thể của $T$ khi $d = 5\;\mathrm{m}$ là bao nhiêu?
\end{problem}
\begin{problem}
{\textbf{\textsc{Roundabout 2}}} Khi khối đi quanh đường tròn, nó được đẩy nhẹ theo phương vuông góc với vận tốc hiện tại của nó và song song với mặt phẳng của đường tròn, gây ra hiện tượng dao động với chu kỳ $T$. Giá trị nhỏ nhất có thể của $T$ khi $d = 5\;\mathrm{m}$ là bao nhiêu?
\end{problem}

\begin{problem}{\textbf{\textsc{Atom Smasher}}}
An alpha particle is the nucleus of a $^4\text{He}$ atom, and is composed of two protons and two neutrons bound together. A neutron is given speed $v$ and collides with an alpha particle at rest. If all five protons and neutrons become unbound as a result of the collision, what is the minimum possible value of $v/c$? You may find the following values useful:
$$m_p = 938.27\text{ MeV}/c^2,\quad m_n = 939.57\text{ MeV}/c^2,\quad m_{\alpha} = 3727.4\text{ MeV}/c^2$$
\end{problem}
\begin{problem}{\textbf{\textsc{Atom Smasher}}}
An alpha particle is the nucleus of a $^4\text{He}$ atom, and is composed of two protons and two neutrons bound together. A neutron is given speed $v$ and collides with an alpha particle at rest. If all five protons and neutrons become unbound as a result of the collision, what is the minimum possible value of $v/c$? You may find the following values useful:
$$m_p = 938.27\text{ MeV}/c^2,\quad m_n = 939.57\text{ MeV}/c^2,\quad m_{\alpha} = 3727.4\text{ MeV}/c^2$$
\end{problem}

\begin{solution}Let the light wave be parameterized as $Ae^{i(kx - \omega t)}$ (i.e the electric field) and let $n = 1 + ia$. Upon entry into the material, observe that $k' = nk$ is the new wave number. Hence the light wave in the fluid is:
$$Ae^{i(nkx - \omega t)} = Ae^{i(kx- \omega t)}e^{-akx}$$
Thus, because intensity is proportional to amplitude squared, we have $I_f = I_0 e^{-2akx}$, giving:
$$x = \frac{1}{2ak}\ln\left(\frac{I_0}{I_f}\right) = \frac{\lambda}{4\pi a}\ln\left(\frac{I_0}{I_f}\right) = \boxed{2.14\;\text{m}}$$
\end{solution}
\begin{solution}Let the light wave be parameterized as $Ae^{i(kx - \omega t)}$ (i.e the electric field) and let $n = 1 + ia$. Upon entry into the material, observe that $k' = nk$ is the new wave number. Hence the light wave in the fluid is:
$$Ae^{i(nkx - \omega t)} = Ae^{i(kx- \omega t)}e^{-akx}$$
Thus, because intensity is proportional to amplitude squared, we have $I_f = I_0 e^{-2akx}$, giving:
$$x = \frac{1}{2ak}\ln\left(\frac{I_0}{I_f}\right) = \frac{\lambda}{4\pi a}\ln\left(\frac{I_0}{I_f}\right) = \boxed{2.14\;\text{m}}$$
\end{solution}

\begin{problem}
\textbf{\textsc{Motorized Pendulum 1}}
A pendulum is made of a massless rod of length $l=0.5000\;\mathrm{m}$ and a point mass $m=15.00\;\mathrm{kg}$ hanging at one end. The angle between the rod and the vertical is $\theta$. A motor attached to the pivot supplies a torque. The maximum value of this torque is angle-dependent and is given by $\tau (\theta)=\frac{1+\cos\theta}{2} \tau_0$ for $0\leq \theta \leq 90^{\circ}$.

\begin{figure}[h]
    \centering
    \includegraphics[width=0.5\linewidth]{problems/figures/mot_pend.png}
    \caption{Motorized pendulum}
    \label{fig:enter-label}
\end{figure}

The pendulum initially is given a small angular velocity counterclockwise and is at $\theta = 0$. The mass is extremely sensitive and cannot tolerate high speeds. Therefore, assume the motor always supplies just enough torque for the mass to move at a negligibly small constant speed. What is the minimum value of $\tau_0$ needed so that the pendulum eventually reaches $\theta=90^{\circ}$?


\end{problem}
\begin{problem}
\textbf{\textsc{Motorized Pendulum 1}}
A pendulum is made of a massless rod of length $l=0.5000\;\mathrm{m}$ and a point mass $m=15.00\;\mathrm{kg}$ hanging at one end. The angle between the rod and the vertical is $\theta$. A motor attached to the pivot supplies a torque. The maximum value of this torque is angle-dependent and is given by $\tau (\theta)=\frac{1+\cos\theta}{2} \tau_0$ for $0\leq \theta \leq 90^{\circ}$.

\begin{figure}[h]
    \centering
    \includegraphics[width=0.5\linewidth]{problems/figures/mot_pend.png}
    \caption{Motorized pendulum}
    \label{fig:enter-label}
\end{figure}

The pendulum initially is given a small angular velocity counterclockwise and is at $\theta = 0$. The mass is extremely sensitive and cannot tolerate high speeds. Therefore, assume the motor always supplies just enough torque for the mass to move at a negligibly small constant speed. What is the minimum value of $\tau_0$ needed so that the pendulum eventually reaches $\theta=90^{\circ}$?


\end{problem}

\begin{problem}
\textbf{\textsc{Motorized Pendulum 2}}
The pendulum initially is at $\theta=0$. This time, the mass is not so sensitive. The motor may supply its full torque for all $\theta$. What is the minimum value of $\tau_0$ needed so that the pendulum reaches $\theta=90^{\circ}$ in a single unidirectional swing?



\end{problem}
    
\begin{problem}
\textbf{\textsc{Motorized Pendulum 2}}
The pendulum initially is at $\theta=0$. This time, the mass is not so sensitive. The motor may supply its full torque for all $\theta$. What is the minimum value of $\tau_0$ needed so that the pendulum reaches $\theta=90^{\circ}$ in a single unidirectional swing?



\end{problem}
    

\begin{problem}
\textbf{\textsc{Motorized Pendulum 3}}
The pendulum initially is at $\theta=0$. The mass contains extremely sensitive electronics that cannot tolerate speeds above $v_{max}=0.1000\;\mathrm{m/s}$. To three significant figures, what is the minimum value of $\tau_0$ needed so that the pendulum reaches $\theta=90^{\circ}$ without exceeding this speed threshold?

\end{problem}
\begin{problem}
\textbf{\textsc{Motorized Pendulum 3}}
The pendulum initially is at $\theta=0$. The mass contains extremely sensitive electronics that cannot tolerate speeds above $v_{max}=0.1000\;\mathrm{m/s}$. To three significant figures, what is the minimum value of $\tau_0$ needed so that the pendulum reaches $\theta=90^{\circ}$ without exceeding this speed threshold?

\end{problem}

\begin{problem}
\textbf{\textsc{Gravitational Oscillations 1}}
%There has been a sad trend in the last decade where more and more people reject basic science and start believing that the earth is flat. While the earth is obviously round, the challenges in making Newtonian gravitation work in a flat world are interesting. So let's have fun at the expense of flat earthers.
You are given the charge distribution on a conductive ellipsoid described by the equation 
\[
\frac{x^2}{a^2} + \frac{y^2}{b^2} + \frac{z^2}{c^2} = 1
\]
If we denote its total charge by \(q\), the surface charge density \(\sigma\) is given by
\begin{equation*}
\sigma = \frac{q}{4\pi abc} \left( \frac{x^2}{a^4} + \frac{y^2}{b^4} + \frac{z^2}{c^4} \right)^{-1 / 2}
\label{ellipsoidCharge}
\end{equation*}
Now suppose we model a planet as a uniform density disk. The issue with this is that everyone at the edges would be pulled towards the center (not ``downwards''). In what follows, suppose that the disk has a fixed radius \(R\) and a height \(h \ll R\). This comes at the cost that different people feel different gravitational ``constants'' downward. Consider the density distribution of the disk \(\rho = \rho(r)\) such that people living on it would only feel a gravitational pull downwards. What is the ratio $\rho(\frac{R}{3})/\rho(\frac{2R}{3})$?

\end{problem}

\begin{problem}
\textbf{\textsc{Gravitational Oscillations 1}}
%There has been a sad trend in the last decade where more and more people reject basic science and start believing that the earth is flat. While the earth is obviously round, the challenges in making Newtonian gravitation work in a flat world are interesting. So let's have fun at the expense of flat earthers.
You are given the charge distribution on a conductive ellipsoid described by the equation 
\[
\frac{x^2}{a^2} + \frac{y^2}{b^2} + \frac{z^2}{c^2} = 1
\]
If we denote its total charge by \(q\), the surface charge density \(\sigma\) is given by
\begin{equation*}
\sigma = \frac{q}{4\pi abc} \left( \frac{x^2}{a^4} + \frac{y^2}{b^4} + \frac{z^2}{c^4} \right)^{-1 / 2}
\label{ellipsoidCharge}
\end{equation*}
Now suppose we model a planet as a uniform density disk. The issue with this is that everyone at the edges would be pulled towards the center (not ``downwards''). In what follows, suppose that the disk has a fixed radius \(R\) and a height \(h \ll R\). This comes at the cost that different people feel different gravitational ``constants'' downward. Consider the density distribution of the disk \(\rho = \rho(r)\) such that people living on it would only feel a gravitational pull downwards. What is the ratio $\rho(\frac{R}{3})/\rho(\frac{2R}{3})$?

\end{problem}


\begin{solution}

Dựa vào công thức \eqref{densityDistribution} chúng ta có thể thấy rằng
$\rho$ sẽ tiến đến vô cùng khi $r \rightarrow R$. Điều này không thực tế, nên hãy xem liệu việc bỏ qua vòng nhỏ ở bên ngoài có giải quyết được vấn đề này hay không.

\begin{gather*}
    \int^{R}_{R - \epsilon} \rho_0 h \frac{2\pi r \mathrm{d} r}{
    \sqrt{1 - \frac{r ^ 2}{R ^ 2}}} = \pi R ^ 2 h \rho_0 \left.\frac{\left( 1
        - \frac{r^2}{R^2} \right)}{\frac{1}{2}}\right|^R_{R - \epsilon} = \\
     = 2 \pi R h \rho_0 \sqrt{2 R \epsilon} \\
     = 4.12973 \cdot 10 ^ {21} \; \mathrm{kg}
\end{gather*}

\end{solution}
\begin{solution}

Dựa vào công thức \eqref{densityDistribution} chúng ta có thể thấy rằng
$\rho$ sẽ tiến đến vô cùng khi $r \rightarrow R$. Điều này không thực tế, nên hãy xem liệu việc bỏ qua vòng nhỏ ở bên ngoài có giải quyết được vấn đề này hay không.

\begin{gather*}
    \int^{R}_{R - \epsilon} \rho_0 h \frac{2\pi r \mathrm{d} r}{
    \sqrt{1 - \frac{r ^ 2}{R ^ 2}}} = \pi R ^ 2 h \rho_0 \left.\frac{\left( 1
        - \frac{r^2}{R^2} \right)}{\frac{1}{2}}\right|^R_{R - \epsilon} = \\
     = 2 \pi R h \rho_0 \sqrt{2 R \epsilon} \\
     = 4.12973 \cdot 10 ^ {21} \; \mathrm{kg}
\end{gather*}

\end{solution}

\begin{solution}

Phần lớn lực hút đến từ khối lượng không bị khoan. Vì vậy, chúng ta chỉ cần xét trường hấp dẫn tại khoảng cách $z$ từ tâm:
Ta có:
\begin{gather*}
 \mathrm{\Gamma} = 2 \pi \mathrm{G} \rho (2 * z) 
\end{gather*}
Khi đó, chuyển động được mô tả bằng:
\begin{gather*}
    \ddot{z} = -4 \pi \mathrm{G} \rho z 
\end{gather*}
Và tần số góc $\omega$ của dao động là
\(\sqrt{4\pi G\rho_{\left( r_0 \right)}}\)
\begin{equation}
    \omega = \sqrt{4\pi G \rho_0} {\left(1 - \frac{r_0^2}{R^2} 
    \right)} ^ {-1 / 4} 
    \label{densityDistribution}
\end{equation}
\begin{gather}
    T= \sqrt{\frac{\pi} {G \rho_0}} {\left(1 - \frac{r_0^2}{R^2} 
    \right)} ^ {1 / 4} \\
    = 2106.6 \;\mathrm{s}
\end{gather}
\end{solution}
\begin{solution}

Phần lớn lực hút đến từ khối lượng không bị khoan. Vì vậy, chúng ta chỉ cần xét trường hấp dẫn tại khoảng cách $z$ từ tâm:
Ta có:
\begin{gather*}
 \mathrm{\Gamma} = 2 \pi \mathrm{G} \rho (2 * z) 
\end{gather*}
Khi đó, chuyển động được mô tả bằng:
\begin{gather*}
    \ddot{z} = -4 \pi \mathrm{G} \rho z 
\end{gather*}
Và tần số góc $\omega$ của dao động là
\(\sqrt{4\pi G\rho_{\left( r_0 \right)}}\)
\begin{equation}
    \omega = \sqrt{4\pi G \rho_0} {\left(1 - \frac{r_0^2}{R^2} 
    \right)} ^ {-1 / 4} 
    \label{densityDistribution}
\end{equation}
\begin{gather}
    T= \sqrt{\frac{\pi} {G \rho_0}} {\left(1 - \frac{r_0^2}{R^2} 
    \right)} ^ {1 / 4} \\
    = 2106.6 \;\mathrm{s}
\end{gather}
\end{solution}

\begin{problem}
{\textbf{\textsc{Thấu kính lỏng}}} Một giếng tròn lớn, với bán kính 1 mét và chiều sâu lớn hơn nhiều so với bán kính ($d \gg r$), được lấp đầy bằng kim loại lỏng phản xạ ánh sáng. Giếng và chất lỏng bên trong nó sau đó được quay xung quanh trục trung tâm với tốc độ góc $\omega = 5\;\mathrm{rad/s}$, làm cho mép bề mặt chất lỏng dâng lên và chạm vào miệng giếng. Được đặt trực tiếp phía trên giếng là một đèn hình tròn với bán kính $1\;\mathrm{m}$, phát ra photon theo chiều dọc xuống dưới với một tỷ lệ và mật độ đồng đều. Nếu tỷ lệ mà các photon rời khỏi đèn là $r$, thì tỷ lệ mà các photon va chạm với kim loại lỏng có thể được biểu diễn dưới dạng $nr$, với $n$ là một hằng số không có đơn vị. Giá trị của $n$ là bao nhiêu?
% \FloatBarrier 
% \begin{figure}[!htbp]
%      \centering
%     \includegraphics[scale=0.42]{problems/figures/wellDiagram.png}
%  \end{figure}
% \FloatBarrier

\end{problem}

\begin{problem}
{\textbf{\textsc{Thấu kính lỏng}}} Một giếng tròn lớn, với bán kính 1 mét và chiều sâu lớn hơn nhiều so với bán kính ($d \gg r$), được lấp đầy bằng kim loại lỏng phản xạ ánh sáng. Giếng và chất lỏng bên trong nó sau đó được quay xung quanh trục trung tâm với tốc độ góc $\omega = 5\;\mathrm{rad/s}$, làm cho mép bề mặt chất lỏng dâng lên và chạm vào miệng giếng. Được đặt trực tiếp phía trên giếng là một đèn hình tròn với bán kính $1\;\mathrm{m}$, phát ra photon theo chiều dọc xuống dưới với một tỷ lệ và mật độ đồng đều. Nếu tỷ lệ mà các photon rời khỏi đèn là $r$, thì tỷ lệ mà các photon va chạm với kim loại lỏng có thể được biểu diễn dưới dạng $nr$, với $n$ là một hằng số không có đơn vị. Giá trị của $n$ là bao nhiêu?
% \FloatBarrier 
% \begin{figure}[!htbp]
%      \centering
%     \includegraphics[scale=0.42]{problems/figures/wellDiagram.png}
%  \end{figure}
% \FloatBarrier

\end{problem}


\begin{problem}{\textbf{\textsc{Trượt điện}}} 
Xem xét một khí gồm các hạt nhỏ, mỗi hạt có điện tích \( q \), bên trong một buồng hình cầu với bán kính $R$. và tâm tại gốc tọa độ. Một trường điện đều $E\hat{\mathbf{x}}$ được áp dụng bên trong buồng. Trường điện được điều chỉnh cho đến khi điểm$(R,0,0)$ có áp suất $P_0$ và tại điểm $(-R,0,0)$ có áp suất $P_0/2$ (ở trạng thái cân bằng). 

Điện trường được giảm nhanh chóng về không và khí đạt đến trạng thái cân bằng một lần nữa. Nếu áp suất cuối cùng trong buồng là $P_1$, tìm tỷ lệ $P_1/P_0$. Bỏ qua các tương tác giữa các hạt và giả sử rằng nhiệt độ của khí vẫn gần như không thay đổi.
\vspace{-0.5cm}
\begin{center}
    \begin{tikzpicture}[scale = 0.75, dot/.style = {circle, fill, minimum size=#1, inner sep=0pt, outer sep=0pt}]
    \draw[very thick,left color=white, right color = white!60!black] (0, 0) circle (3);
    \draw[very thick] (2.9,0)--(3.1,0);
    \node at (3, 0) [right]{$P_0$};
    \draw[very thick] (-3.1,0)--(-2.9,0);
    \node at (-3.05,0) [left]{$P_0/2$};
    \node at (2,-0.5) [dot=4]{};
    \node at (-0.5, -1) [dot=4]{};
    \node at (1, 2) [dot=4]{};
    \node at (0, 1)[dot=4]{};
    \node at (0.75, -0.75)[dot=4]{};
    \node at (-2, -0.75)[dot=4]{};
    \node at (-1.5, 1) [dot=4]{};
    \node at (1.5, -2) [dot=4]{};
    \node at (-0.5, 2.25) [dot=4]{};
    \node at (-0.25, -2.25) [dot=4]{};
    \node at (2.25, 1) [dot=4]{};
    \node at (-1.5, 1) [above left]{$q$};
    \node at (2.4, 2.4) {$R$};
    \draw [->] (-0.5, -3.5) -- (0.5, -3.5) node[right]{$E$};
    \end{tikzpicture}
\end{center}
\end{problem}

\begin{problem}{\textbf{\textsc{Trượt điện}}} 
Xem xét một khí gồm các hạt nhỏ, mỗi hạt có điện tích \( q \), bên trong một buồng hình cầu với bán kính $R$. và tâm tại gốc tọa độ. Một trường điện đều $E\hat{\mathbf{x}}$ được áp dụng bên trong buồng. Trường điện được điều chỉnh cho đến khi điểm$(R,0,0)$ có áp suất $P_0$ và tại điểm $(-R,0,0)$ có áp suất $P_0/2$ (ở trạng thái cân bằng). 

Điện trường được giảm nhanh chóng về không và khí đạt đến trạng thái cân bằng một lần nữa. Nếu áp suất cuối cùng trong buồng là $P_1$, tìm tỷ lệ $P_1/P_0$. Bỏ qua các tương tác giữa các hạt và giả sử rằng nhiệt độ của khí vẫn gần như không thay đổi.
\vspace{-0.5cm}
\begin{center}
    \begin{tikzpicture}[scale = 0.75, dot/.style = {circle, fill, minimum size=#1, inner sep=0pt, outer sep=0pt}]
    \draw[very thick,left color=white, right color = white!60!black] (0, 0) circle (3);
    \draw[very thick] (2.9,0)--(3.1,0);
    \node at (3, 0) [right]{$P_0$};
    \draw[very thick] (-3.1,0)--(-2.9,0);
    \node at (-3.05,0) [left]{$P_0/2$};
    \node at (2,-0.5) [dot=4]{};
    \node at (-0.5, -1) [dot=4]{};
    \node at (1, 2) [dot=4]{};
    \node at (0, 1)[dot=4]{};
    \node at (0.75, -0.75)[dot=4]{};
    \node at (-2, -0.75)[dot=4]{};
    \node at (-1.5, 1) [dot=4]{};
    \node at (1.5, -2) [dot=4]{};
    \node at (-0.5, 2.25) [dot=4]{};
    \node at (-0.25, -2.25) [dot=4]{};
    \node at (2.25, 1) [dot=4]{};
    \node at (-1.5, 1) [above left]{$q$};
    \node at (2.4, 2.4) {$R$};
    \draw [->] (-0.5, -3.5) -- (0.5, -3.5) node[right]{$E$};
    \end{tikzpicture}
\end{center}
\end{problem}


\begin{solution}
We first find the optimal speed to reach a point $(X,Y)$ in space from the origin $(0,0).$ If the launching angle is $\alpha$ and the launch speed $v$, the kinematic equations for the $x$- and $y$-directions are 
\begin{align*}
    x&=vt\cos\alpha & y&=-\frac 12 gt^2+vt\sin\alpha.
\end{align*}
Solving for the shape of the trajectory gives
\[y=-\frac{gx^2}{2v^2\cos^2\alpha}+x\tan\alpha.\]
If $v$ is large enough, there exists a respective $\alpha$ for which the point $(x,y)=(X,Y)$ satisfies the equation. Now we note that $1/\cos^2\alpha=1+\tan^2\alpha$ which turns the trajectory into a quadratic in $\xi=\tan\alpha$
\[\xi^2-\frac{2v^2}{gd}\xi+\frac{2v^2h}{gd^2}+1=0.\]
This has real solutions for $\xi$ if the discriminant is non-negative. If the discriminant is positive, it means there are two angles for the respective speed. This clearly means that the speed is not optimal. Hence we want the discriminant to be zero:
\[\frac{v^4}{g^2d^2}-\frac{2v^2h}{gd^2}-1=0.\]
This is a biquadratic in $v$ and solving:
\[v^2=gY+g\sqrt{X^2+Y^2}\implies v_0=\sqrt{gY+g\sqrt{X^2+Y^2}}.\]

Now, let's get back to the problem at hand. Let $x$ be the distance from the edge of the table at the bouncing point. As the speed is conserved during the bounce, the minimal speed must be such that at the point $x$ the ball can reach both the edge of the table (kinematic trajectories are reversible) and the top of the net (the ball has to go over the net). Using the derived result
\[v(x)=\max\left\{\sqrt{gh+g\sqrt{h^2+(d-x)^2}},\sqrt{gx}\right\}.\]
Clearly the first term increases with $x$ and the other term decreases with $x$. Thus the global minimum of $v(x)$ is found at the point where the two terms are equal. This corresponds to
\[h+\sqrt{h^2+(d-x)^2}=x\implies x=\frac{d^2}{2(d-h)}\]
and thus
\[v_1=d\sqrt{\frac{g}{2(d-h)}}\approx2.75\;\mathrm{m/s}.\]
\end{solution}
\begin{solution}
We first find the optimal speed to reach a point $(X,Y)$ in space from the origin $(0,0).$ If the launching angle is $\alpha$ and the launch speed $v$, the kinematic equations for the $x$- and $y$-directions are 
\begin{align*}
    x&=vt\cos\alpha & y&=-\frac 12 gt^2+vt\sin\alpha.
\end{align*}
Solving for the shape of the trajectory gives
\[y=-\frac{gx^2}{2v^2\cos^2\alpha}+x\tan\alpha.\]
If $v$ is large enough, there exists a respective $\alpha$ for which the point $(x,y)=(X,Y)$ satisfies the equation. Now we note that $1/\cos^2\alpha=1+\tan^2\alpha$ which turns the trajectory into a quadratic in $\xi=\tan\alpha$
\[\xi^2-\frac{2v^2}{gd}\xi+\frac{2v^2h}{gd^2}+1=0.\]
This has real solutions for $\xi$ if the discriminant is non-negative. If the discriminant is positive, it means there are two angles for the respective speed. This clearly means that the speed is not optimal. Hence we want the discriminant to be zero:
\[\frac{v^4}{g^2d^2}-\frac{2v^2h}{gd^2}-1=0.\]
This is a biquadratic in $v$ and solving:
\[v^2=gY+g\sqrt{X^2+Y^2}\implies v_0=\sqrt{gY+g\sqrt{X^2+Y^2}}.\]

Now, let's get back to the problem at hand. Let $x$ be the distance from the edge of the table at the bouncing point. As the speed is conserved during the bounce, the minimal speed must be such that at the point $x$ the ball can reach both the edge of the table (kinematic trajectories are reversible) and the top of the net (the ball has to go over the net). Using the derived result
\[v(x)=\max\left\{\sqrt{gh+g\sqrt{h^2+(d-x)^2}},\sqrt{gx}\right\}.\]
Clearly the first term increases with $x$ and the other term decreases with $x$. Thus the global minimum of $v(x)$ is found at the point where the two terms are equal. This corresponds to
\[h+\sqrt{h^2+(d-x)^2}=x\implies x=\frac{d^2}{2(d-h)}\]
and thus
\[v_1=d\sqrt{\frac{g}{2(d-h)}}\approx2.75\;\mathrm{m/s}.\]
\end{solution}

\begin{solution}
We can generalise the argument of the last problem by noting that
\[v_n\geq\sqrt{gh+g\sqrt{h^2+x^2}},\]
where $x$ is the distance of the last bounce from the net. Thus we wish to get the ball as close to the net as possible in $n$ bounces. This happens when all the previous bouncers happen at an angle $\ang{45}$. The range of one of these bounces is $\ell=\frac{v^2}{g}$ as one can find from the trajectory equation. Thus we have that the distance of the final bounce from the net is:
\[x=d-n\ell.\]
We also note that $v_n\geq \sqrt{2gh}$ based on the lower limit of the previous equation ($x=0$). So if we reach the net in less than $n$ bounces and thus get to $x=0$ for the last bounce, the minimal speed will still be $v_n$. I.e. when $d-n\ell < 0 \implies n>\frac{d}{2h}$ the minimal speed will be
\[v_n=\sqrt{2gh}.\]
As $d/2h\approx 4.5$, $N=5.$\\

Now if $n<N$, we won't reach the net in $n$ bounces, and the distance from the net for the last bounce is $x=d-n\ell.$ I.e. the speed to get the ball's last bounce to be a distance $x$ away will be $v=\sqrt{g(d-x)/n}.$ Similarly to the previous problems, this means that the optimal speed is achieved at the $x$ for which the optimal speed to get there is the same as to go over the net from there. I.e. we get that
\[h+\sqrt{h^2+x^2}=\frac{d-x}{n},\]
which yields the quadratic equation
\[(n^2-1)x^2+2(d-nh)x+2ndh-d^2=0,\]
from which we get
\[x=\frac{nh-d\pm n\sqrt{h^2+d^2-2ndh}}{n^2-1}.\]
As $d>2nh$ the smaller root is negative and thus non-physical. Hence, we take the positive root. Substituting this in one of the expressions for $v$ we get
\[v_n=\sqrt{\frac{g}{n^2-1}\left(nd-h-\sqrt{h^2+d^2-2ndh}\right)}.\]

Thus
\[v_{N-1}^N=v_4^5\approx\boxed{17.9\;(\mathrm{m/s})^5.}\]
\end{solution}
\begin{solution}
We can generalise the argument of the last problem by noting that
\[v_n\geq\sqrt{gh+g\sqrt{h^2+x^2}},\]
where $x$ is the distance of the last bounce from the net. Thus we wish to get the ball as close to the net as possible in $n$ bounces. This happens when all the previous bouncers happen at an angle $\ang{45}$. The range of one of these bounces is $\ell=\frac{v^2}{g}$ as one can find from the trajectory equation. Thus we have that the distance of the final bounce from the net is:
\[x=d-n\ell.\]
We also note that $v_n\geq \sqrt{2gh}$ based on the lower limit of the previous equation ($x=0$). So if we reach the net in less than $n$ bounces and thus get to $x=0$ for the last bounce, the minimal speed will still be $v_n$. I.e. when $d-n\ell < 0 \implies n>\frac{d}{2h}$ the minimal speed will be
\[v_n=\sqrt{2gh}.\]
As $d/2h\approx 4.5$, $N=5.$\\

Now if $n<N$, we won't reach the net in $n$ bounces, and the distance from the net for the last bounce is $x=d-n\ell.$ I.e. the speed to get the ball's last bounce to be a distance $x$ away will be $v=\sqrt{g(d-x)/n}.$ Similarly to the previous problems, this means that the optimal speed is achieved at the $x$ for which the optimal speed to get there is the same as to go over the net from there. I.e. we get that
\[h+\sqrt{h^2+x^2}=\frac{d-x}{n},\]
which yields the quadratic equation
\[(n^2-1)x^2+2(d-nh)x+2ndh-d^2=0,\]
from which we get
\[x=\frac{nh-d\pm n\sqrt{h^2+d^2-2ndh}}{n^2-1}.\]
As $d>2nh$ the smaller root is negative and thus non-physical. Hence, we take the positive root. Substituting this in one of the expressions for $v$ we get
\[v_n=\sqrt{\frac{g}{n^2-1}\left(nd-h-\sqrt{h^2+d^2-2ndh}\right)}.\]

Thus
\[v_{N-1}^N=v_4^5\approx\boxed{17.9\;(\mathrm{m/s})^5.}\]
\end{solution}

\begin{solution}
The point light source emits light with spherical symmetry, but because of the cylindrical housing what is interpreted is a light ``cone'' instead. This cone intersecting the wall at $x=-D$ is what gives rise to the hyperbola shown.

\begin{center}
    \includegraphics[height=0.35\textwidth]{solutions/figures/lightConeAnswerDiagram.png}
    \hfill
    \includegraphics[height=0.35\textwidth]{solutions/figures/lightConeAnswerGraph.png}
\end{center}

By virtue of \href{https://www.youtube.com/watch?v=kfWDkVct5mM}{conic sections} having constant eccentricity, it can be expressed as: $e = \sin\theta_\text{plane}/\sin\theta_\text{cone} = 1/\sin\theta_\text{cone}$ since the wall has $\theta = 90^\circ$. Alternatively, one can also express eccentricity by the formula $e=c/a=\sqrt{a^2+b^2}/a$ where $c=\sqrt{a^2+b^2}$ is the distance from the origin to the focus. Rearranging this gives us:

$$e = \frac{1}{\sin\theta_\text{cone}} = \sqrt{1+\left(\frac{b}{a}\right)^2} \implies \tan\theta_\text{cone} = \frac{a}{b} = 1.33$$

This can then be related to the horizontal distance $D$ by $\tan\theta_\text{cone} = a/D$ to give $D = b = 1.5\text{m}/(4/3) = \boxed{1.125\;\text{m}}$.
\end{solution}
\begin{solution}
The point light source emits light with spherical symmetry, but because of the cylindrical housing what is interpreted is a light ``cone'' instead. This cone intersecting the wall at $x=-D$ is what gives rise to the hyperbola shown.

\begin{center}
    \includegraphics[height=0.35\textwidth]{solutions/figures/lightConeAnswerDiagram.png}
    \hfill
    \includegraphics[height=0.35\textwidth]{solutions/figures/lightConeAnswerGraph.png}
\end{center}

By virtue of \href{https://www.youtube.com/watch?v=kfWDkVct5mM}{conic sections} having constant eccentricity, it can be expressed as: $e = \sin\theta_\text{plane}/\sin\theta_\text{cone} = 1/\sin\theta_\text{cone}$ since the wall has $\theta = 90^\circ$. Alternatively, one can also express eccentricity by the formula $e=c/a=\sqrt{a^2+b^2}/a$ where $c=\sqrt{a^2+b^2}$ is the distance from the origin to the focus. Rearranging this gives us:

$$e = \frac{1}{\sin\theta_\text{cone}} = \sqrt{1+\left(\frac{b}{a}\right)^2} \implies \tan\theta_\text{cone} = \frac{a}{b} = 1.33$$

This can then be related to the horizontal distance $D$ by $\tan\theta_\text{cone} = a/D$ to give $D = b = 1.5\text{m}/(4/3) = \boxed{1.125\;\text{m}}$.
\end{solution}

% \begin{solution}
%     The circle below the lamp has a radius of 

% $$r = \sqrt{\frac{3\pi}{\pi}} = \sqrt{3}$$

% The greatest angle that the light travels from the vertical is therefore

% $$\theta = \arctan{\frac{\sqrt{3}}{3}} = 30^{\circ}$$

% We can examine a cross section of the light's path. Consider Snell's Law, $n_1\sin{\theta} = n_2\sin{\theta}$. It can be seen that at any point along the light's path through the liquid, $n\sin{\theta}$ will always equal some constant value. In this case, because the incident angle is 30 degrees and the index of refraction of air is 1, this constant value will be $\sin{30^{\circ}} = 0.5$. We can assign the light ray the initial conditions $(x,y) = (\sqrt{3},0)$, as the light enters the Ophonium a horizontal distance of $\sqrt{3}$ from the lamp.

% We have:

% $$n\sin{\theta} = \frac{1}{2}\longrightarrow \sin{\theta} = \frac{1}{2n} = \frac{1}{2+4y}$$

% $$\cos\bigg(\frac{\pi}{2}-\theta\bigg) = \frac{1}{2+4y}\longrightarrow\frac{1}{\sqrt{1+\tan^2\big(\frac{\pi}{2}-\theta\big)}} = \frac{1}{2+4y}$$

% $$\frac{1}{\sqrt{1+\big(\frac{\delta y}{\delta x}\big)^2}} = \frac{1}{2+4y}\longrightarrow \sqrt{1+\big(\frac{\delta y}{\delta x}\big)^2} = 2 + 4y$$

% $$\frac{\delta y}{\delta x} = \sqrt{(2+4y)^2 - 1}\longrightarrow \frac{\delta x}{\delta y} = \frac{1}{\sqrt{(2+4y)^2 - 1}}$$

% $$\int \delta x = \int\frac{1}{\sqrt{(2+4y)^2 - 1}}\delta y$$

% $$x = \frac{1}{4}\cosh^{-1}(2+4y)+ C \longrightarrow x = \frac{1}{4}\bigg(\cosh^{-1}(2+4y) + 4\sqrt{3} - \ln(2 + \sqrt{3})\bigg)$$

% Plugging in $y = 5$ gives $x \approx 2.3487$. This is the radius of the circle formed at the bottom of the vat. Thus, the total area is

% $$x^2\pi \approx \boxed{17.33\;\mathrm{m^2}}$$
% \end{solution}

\begin{solution}
    Vòng tròn dưới đèn có bán kính là 

$$r = \sqrt{\frac{3\pi}{\pi}} = \sqrt{3}$$

Góc lớn nhất mà ánh sáng di chuyển từ phương thẳng đứng do đó là

$$\theta = \arctan{\frac{\sqrt{3}}{3}} = 30^{\circ}$$

Chúng ta có thể kiểm tra một mặt cắt của đường đi của ánh sáng. Xem xét Định luật Snell, $n_1\sin{\theta} = n_2\sin{\theta}$. Có thể thấy rằng tại bất kỳ điểm nào dọc theo đường đi của ánh sáng qua chất lỏng, $n\sin{\theta}$ sẽ luôn bằng một giá trị không đổi. Trong trường hợp này, vì góc tới là 30 độ và chiết suất của không khí là 1, giá trị không đổi này sẽ là $\sin{30^{\circ}} = 0.5$. Chúng ta có thể gán cho tia sáng các điều kiện ban đầu $(x,y) = (\sqrt{3},0)$, khi ánh sáng đi vào Ophonium một khoảng ngang $\sqrt{3}$ từ đèn.

Chúng ta có:

$$n\sin{\theta} = \frac{1}{2}\longrightarrow \sin{\theta} = \frac{1}{2n} = \frac{1}{2+4y}$$

$$\cos\bigg(\frac{\pi}{2}-\theta\bigg) = \frac{1}{2+4y}\longrightarrow\frac{1}{\sqrt{1+\tan^2\big(\frac{\pi}{2}-\theta\big)}} = \frac{1}{2+4y}$$

$$\frac{1}{\sqrt{1+\big(\frac{\delta y}{\delta x}\big)^2}} = \frac{1}{2+4y}\longrightarrow \sqrt{1+\big(\frac{\delta y}{\delta x}\big)^2} = 2 + 4y$$

$$\frac{\delta y}{\delta x} = \sqrt{(2+4y)^2 - 1}\longrightarrow \frac{\delta x}{\delta y} = \frac{1}{\sqrt{(2+4y)^2 - 1}}$$

$$\int \delta x = \int\frac{1}{\sqrt{(2+4y)^2 - 1}}\delta y$$

$$x = \frac{1}{4}\cosh^{-1}(2+4y)+ C \longrightarrow x = \frac{1}{4}\bigg(\cosh^{-1}(2+4y) + 4\sqrt{3} - \ln(2 + \sqrt{3})\bigg)$$

Thay $y = 5$ vào ta có $x \approx 2.3487$. Đây là bán kính của vòng tròn được tạo ra ở đáy thùng. Do đó, tổng diện tích là

$$x^2\pi \approx \boxed{17.33\;\mathrm{m^2}}$$
\end{solution}

% \begin{solution}
%     The circle below the lamp has a radius of 

% $$r = \sqrt{\frac{3\pi}{\pi}} = \sqrt{3}$$

% The greatest angle that the light travels from the vertical is therefore

% $$\theta = \arctan{\frac{\sqrt{3}}{3}} = 30^{\circ}$$

% We can examine a cross section of the light's path. Consider Snell's Law, $n_1\sin{\theta} = n_2\sin{\theta}$. It can be seen that at any point along the light's path through the liquid, $n\sin{\theta}$ will always equal some constant value. In this case, because the incident angle is 30 degrees and the index of refraction of air is 1, this constant value will be $\sin{30^{\circ}} = 0.5$. We can assign the light ray the initial conditions $(x,y) = (\sqrt{3},0)$, as the light enters the Ophonium a horizontal distance of $\sqrt{3}$ from the lamp.

% We have:

% $$n\sin{\theta} = \frac{1}{2}\longrightarrow \sin{\theta} = \frac{1}{2n} = \frac{1}{2+4y}$$

% $$\cos\bigg(\frac{\pi}{2}-\theta\bigg) = \frac{1}{2+4y}\longrightarrow\frac{1}{\sqrt{1+\tan^2\big(\frac{\pi}{2}-\theta\big)}} = \frac{1}{2+4y}$$

% $$\frac{1}{\sqrt{1+\big(\frac{\delta y}{\delta x}\big)^2}} = \frac{1}{2+4y}\longrightarrow \sqrt{1+\big(\frac{\delta y}{\delta x}\big)^2} = 2 + 4y$$

% $$\frac{\delta y}{\delta x} = \sqrt{(2+4y)^2 - 1}\longrightarrow \frac{\delta x}{\delta y} = \frac{1}{\sqrt{(2+4y)^2 - 1}}$$

% $$\int \delta x = \int\frac{1}{\sqrt{(2+4y)^2 - 1}}\delta y$$

% $$x = \frac{1}{4}\cosh^{-1}(2+4y)+ C \longrightarrow x = \frac{1}{4}\bigg(\cosh^{-1}(2+4y) + 4\sqrt{3} - \ln(2 + \sqrt{3})\bigg)$$

% Plugging in $y = 5$ gives $x \approx 2.3487$. This is the radius of the circle formed at the bottom of the vat. Thus, the total area is

% $$x^2\pi \approx \boxed{17.33\;\mathrm{m^2}}$$
% \end{solution}

\begin{solution}
    Vòng tròn dưới đèn có bán kính là 

$$r = \sqrt{\frac{3\pi}{\pi}} = \sqrt{3}$$

Góc lớn nhất mà ánh sáng di chuyển từ phương thẳng đứng do đó là

$$\theta = \arctan{\frac{\sqrt{3}}{3}} = 30^{\circ}$$

Chúng ta có thể kiểm tra một mặt cắt của đường đi của ánh sáng. Xem xét Định luật Snell, $n_1\sin{\theta} = n_2\sin{\theta}$. Có thể thấy rằng tại bất kỳ điểm nào dọc theo đường đi của ánh sáng qua chất lỏng, $n\sin{\theta}$ sẽ luôn bằng một giá trị không đổi. Trong trường hợp này, vì góc tới là 30 độ và chiết suất của không khí là 1, giá trị không đổi này sẽ là $\sin{30^{\circ}} = 0.5$. Chúng ta có thể gán cho tia sáng các điều kiện ban đầu $(x,y) = (\sqrt{3},0)$, khi ánh sáng đi vào Ophonium một khoảng ngang $\sqrt{3}$ từ đèn.

Chúng ta có:

$$n\sin{\theta} = \frac{1}{2}\longrightarrow \sin{\theta} = \frac{1}{2n} = \frac{1}{2+4y}$$

$$\cos\bigg(\frac{\pi}{2}-\theta\bigg) = \frac{1}{2+4y}\longrightarrow\frac{1}{\sqrt{1+\tan^2\big(\frac{\pi}{2}-\theta\big)}} = \frac{1}{2+4y}$$

$$\frac{1}{\sqrt{1+\big(\frac{\delta y}{\delta x}\big)^2}} = \frac{1}{2+4y}\longrightarrow \sqrt{1+\big(\frac{\delta y}{\delta x}\big)^2} = 2 + 4y$$

$$\frac{\delta y}{\delta x} = \sqrt{(2+4y)^2 - 1}\longrightarrow \frac{\delta x}{\delta y} = \frac{1}{\sqrt{(2+4y)^2 - 1}}$$

$$\int \delta x = \int\frac{1}{\sqrt{(2+4y)^2 - 1}}\delta y$$

$$x = \frac{1}{4}\cosh^{-1}(2+4y)+ C \longrightarrow x = \frac{1}{4}\bigg(\cosh^{-1}(2+4y) + 4\sqrt{3} - \ln(2 + \sqrt{3})\bigg)$$

Thay $y = 5$ vào ta có $x \approx 2.3487$. Đây là bán kính của vòng tròn được tạo ra ở đáy thùng. Do đó, tổng diện tích là

$$x^2\pi \approx \boxed{17.33\;\mathrm{m^2}}$$
\end{solution}


% \begin{solution}
% Clearly (as $d>h$) if $v$ is high enough, the optimal trajectory to reach $x_{\max}$ is achieved when the launch angle $\alpha=\ang{45}$. Thus we have to first determine whether or not $v$ is high enough.\\

% If $\alpha=\ang{45}$ corresponds to the optimal trajectory over the wall, the highest point of the wall $(d,h)$ is at or under the trajectory. The equation of the trajectory is well-known and easy to derive:
% \[y=-\frac{gx^2}{2v^2\cos^2\alpha}+x\tan\alpha\]
% so $\alpha=\ang{45}$ is optimal if
% \[h\leq -\frac{gd^2}{v^2}+d \implies v\geq d\sqrt{\frac{g}{d-h}}=14\;\mathrm{m/s},\]
% which is not true. Thus $\alpha=\ang{45}$ is not optimal.\\

% Now clearly the ball cannot go over the wall if $\alpha\leq\ang{45}$. Thus we have to increase the angle which decreases the range as the range is a decreasing function of $\alpha$ when $\alpha>\ang{45}$ which can be seen from the range equation (comes directly from the trajectory equation above):
% \[R=\frac{2v^2\sin\alpha\cos\alpha}{g}=\frac{v^2\sin2\alpha}{g}.\]

% Thus the optimal trajectory now is the one that just barely touches the top of the wall for the minimal $\alpha$. I.e. we wish $(d,h)$ to be on the trajectory. If we plug in $(x,y)=(d,h)$ to the trajectory equation and use $1/\cos^2\alpha=1+\tan^2\alpha$ and solve for $\xi\equiv\tan\alpha$ from the quadratic equation that ensues we get two possible roots
% \[\xi_{\pm}=\frac{v^2}{gd}\pm\sqrt{\frac{v^4}{g^2d^2}-\frac{2v^2h}{gd^2}-1}.\]
% As $\xi=\tan\alpha$ and $\tan$ is an increasing function (in our range), we are interested in the smaller root. Thus
% \[\alpha=\arctan{\xi_-}.\]
% Thus the range equation gives us
% \[x_{\max}=\frac{2\xi_-}{1+\xi_-^2}\frac{v^2}{g}.\]

% The argument for the minimal distance is essentially the same. The top of the wall still has to be under the trajectory and $\alpha>\ang{45}$ so now we want the maximal $\alpha$ which corresponds to $\xi_+.$ Thus
% \[x_{\min}=\frac{2\xi_+}{1+\xi_+^2}\frac{v^2}{g}.\]
% And hence
% \[x_{\max}-x_{\min}=\frac{2v^2}{g}\left(\frac{\xi_-}{1+\xi_-^2}-\frac{\xi_+}{1+\xi_+^2}\right)\approx\boxed{5.38\;\mathrm{m}}.\]

% Note for a full solution (not necessary to get the numerical answer) one should also check if the launch speed given is even high enough to go over the wall at all. This minimal speed is relatively easy to derive (especially using the envelope curve):
% \[v_{\min}=\sqrt{gh+g\sqrt{h^2+d^2}}\approx{12.6}\;\mathrm{m/s}<v,\]
% so the question is well-posed.
% \end{solution}

\begin{solution}
Rõ ràng (vì $d>h$) nếu $v$ đủ lớn, quỹ đạo tối ưu để đạt $x_{\max}$ đạt được khi góc phóng $\alpha=\ang{45}$. Do đó, chúng ta phải xác định trước liệu $v$ có đủ lớn hay không.\\

Nếu $\alpha=\ang{45}$ tương ứng với quỹ đạo tối ưu qua tường, điểm cao nhất của tường $(d,h)$ nằm trên hoặc dưới quỹ đạo. Phương trình của quỹ đạo đã được biết đến và dễ dàng suy ra:
\[y=-\frac{gx^2}{2v^2\cos^2\alpha}+x\tan\alpha\]
vì vậy $\alpha=\ang{45}$ là tối ưu nếu
\[h\leq -\frac{gd^2}{v^2}+d \implies v\geq d\sqrt{\frac{g}{d-h}}=14\;\mathrm{m/s},\]
điều này không đúng. Do đó $\alpha=\ang{45}$ không phải là tối ưu.\\

Bây giờ rõ ràng quả bóng không thể vượt qua tường nếu $\alpha\leq\ang{45}$. Do đó, chúng ta phải tăng góc phóng, điều này làm giảm tầm xa vì tầm xa là một hàm giảm của $\alpha$ khi $\alpha>\ang{45}$, điều này có thể thấy từ phương trình tầm xa (trực tiếp từ phương trình quỹ đạo ở trên):
\[R=\frac{2v^2\sin\alpha\cos\alpha}{g}=\frac{v^2\sin2\alpha}{g}.\]

Do đó, quỹ đạo tối ưu bây giờ là quỹ đạo chỉ vừa chạm đỉnh của tường với $\alpha$ nhỏ nhất. Tức là chúng ta muốn $(d,h)$ nằm trên quỹ đạo. Nếu chúng ta thay $(x,y)=(d,h)$ vào phương trình quỹ đạo và sử dụng $1/\cos^2\alpha=1+\tan^2\alpha$ và giải cho $\xi\equiv\tan\alpha$ từ phương trình bậc hai, chúng ta có hai nghiệm khả dĩ
\[\xi_{\pm}=\frac{v^2}{gd}\pm\sqrt{\frac{v^4}{g^2d^2}-\frac{2v^2h}{gd^2}-1}.\]
Vì $\xi=\tan\alpha$ và $\tan$ là một hàm tăng (trong phạm vi của chúng ta), chúng ta quan tâm đến nghiệm nhỏ hơn. Do đó
\[\alpha=\arctan{\xi_-}.\]
Do đó, phương trình tầm xa cho chúng ta
\[x_{\max}=\frac{2\xi_-}{1+\xi_-^2}\frac{v^2}{g}.\]

Lập luận cho khoảng cách tối thiểu về cơ bản là giống nhau. Đỉnh của tường vẫn phải nằm dưới quỹ đạo và $\alpha>\ang{45}$ nên bây giờ chúng ta muốn $\alpha$ lớn nhất tương ứng với $\xi_+.$ Do đó
\[x_{\min}=\frac{2\xi_+}{1+\xi_+^2}\frac{v^2}{g}.\]
Và do đó
\[x_{\max}-x_{\min}=\frac{2v^2}{g}\left(\frac{\xi_-}{1+\xi_-^2}-\frac{\xi_+}{1+\xi_+^2}\right)\approx\boxed{5.38\;\mathrm{m}}.\]

Lưu ý để có một giải pháp đầy đủ (không cần thiết để có được câu trả lời số) người ta cũng nên kiểm tra xem vận tốc phóng đã cho có đủ lớn để vượt qua tường hay không. Vận tốc tối thiểu này tương đối dễ suy ra (đặc biệt là sử dụng đường bao):
\[v_{\min}=\sqrt{gh+g\sqrt{h^2+d^2}}\approx{12.6}\;\mathrm{m/s}<v,\]
vì vậy câu hỏi được đặt ra là hợp lý.
\end{solution}

% \begin{solution}
% Clearly (as $d>h$) if $v$ is high enough, the optimal trajectory to reach $x_{\max}$ is achieved when the launch angle $\alpha=\ang{45}$. Thus we have to first determine whether or not $v$ is high enough.\\

% If $\alpha=\ang{45}$ corresponds to the optimal trajectory over the wall, the highest point of the wall $(d,h)$ is at or under the trajectory. The equation of the trajectory is well-known and easy to derive:
% \[y=-\frac{gx^2}{2v^2\cos^2\alpha}+x\tan\alpha\]
% so $\alpha=\ang{45}$ is optimal if
% \[h\leq -\frac{gd^2}{v^2}+d \implies v\geq d\sqrt{\frac{g}{d-h}}=14\;\mathrm{m/s},\]
% which is not true. Thus $\alpha=\ang{45}$ is not optimal.\\

% Now clearly the ball cannot go over the wall if $\alpha\leq\ang{45}$. Thus we have to increase the angle which decreases the range as the range is a decreasing function of $\alpha$ when $\alpha>\ang{45}$ which can be seen from the range equation (comes directly from the trajectory equation above):
% \[R=\frac{2v^2\sin\alpha\cos\alpha}{g}=\frac{v^2\sin2\alpha}{g}.\]

% Thus the optimal trajectory now is the one that just barely touches the top of the wall for the minimal $\alpha$. I.e. we wish $(d,h)$ to be on the trajectory. If we plug in $(x,y)=(d,h)$ to the trajectory equation and use $1/\cos^2\alpha=1+\tan^2\alpha$ and solve for $\xi\equiv\tan\alpha$ from the quadratic equation that ensues we get two possible roots
% \[\xi_{\pm}=\frac{v^2}{gd}\pm\sqrt{\frac{v^4}{g^2d^2}-\frac{2v^2h}{gd^2}-1}.\]
% As $\xi=\tan\alpha$ and $\tan$ is an increasing function (in our range), we are interested in the smaller root. Thus
% \[\alpha=\arctan{\xi_-}.\]
% Thus the range equation gives us
% \[x_{\max}=\frac{2\xi_-}{1+\xi_-^2}\frac{v^2}{g}.\]

% The argument for the minimal distance is essentially the same. The top of the wall still has to be under the trajectory and $\alpha>\ang{45}$ so now we want the maximal $\alpha$ which corresponds to $\xi_+.$ Thus
% \[x_{\min}=\frac{2\xi_+}{1+\xi_+^2}\frac{v^2}{g}.\]
% And hence
% \[x_{\max}-x_{\min}=\frac{2v^2}{g}\left(\frac{\xi_-}{1+\xi_-^2}-\frac{\xi_+}{1+\xi_+^2}\right)\approx\boxed{5.38\;\mathrm{m}}.\]

% Note for a full solution (not necessary to get the numerical answer) one should also check if the launch speed given is even high enough to go over the wall at all. This minimal speed is relatively easy to derive (especially using the envelope curve):
% \[v_{\min}=\sqrt{gh+g\sqrt{h^2+d^2}}\approx{12.6}\;\mathrm{m/s}<v,\]
% so the question is well-posed.
% \end{solution}

\begin{solution}
Rõ ràng (vì $d>h$) nếu $v$ đủ lớn, quỹ đạo tối ưu để đạt $x_{\max}$ đạt được khi góc phóng $\alpha=\ang{45}$. Do đó, chúng ta phải xác định trước liệu $v$ có đủ lớn hay không.\\

Nếu $\alpha=\ang{45}$ tương ứng với quỹ đạo tối ưu qua tường, điểm cao nhất của tường $(d,h)$ nằm trên hoặc dưới quỹ đạo. Phương trình của quỹ đạo đã được biết đến và dễ dàng suy ra:
\[y=-\frac{gx^2}{2v^2\cos^2\alpha}+x\tan\alpha\]
vì vậy $\alpha=\ang{45}$ là tối ưu nếu
\[h\leq -\frac{gd^2}{v^2}+d \implies v\geq d\sqrt{\frac{g}{d-h}}=14\;\mathrm{m/s},\]
điều này không đúng. Do đó $\alpha=\ang{45}$ không phải là tối ưu.\\

Bây giờ rõ ràng quả bóng không thể vượt qua tường nếu $\alpha\leq\ang{45}$. Do đó, chúng ta phải tăng góc phóng, điều này làm giảm tầm xa vì tầm xa là một hàm giảm của $\alpha$ khi $\alpha>\ang{45}$, điều này có thể thấy từ phương trình tầm xa (trực tiếp từ phương trình quỹ đạo ở trên):
\[R=\frac{2v^2\sin\alpha\cos\alpha}{g}=\frac{v^2\sin2\alpha}{g}.\]

Do đó, quỹ đạo tối ưu bây giờ là quỹ đạo chỉ vừa chạm đỉnh của tường với $\alpha$ nhỏ nhất. Tức là chúng ta muốn $(d,h)$ nằm trên quỹ đạo. Nếu chúng ta thay $(x,y)=(d,h)$ vào phương trình quỹ đạo và sử dụng $1/\cos^2\alpha=1+\tan^2\alpha$ và giải cho $\xi\equiv\tan\alpha$ từ phương trình bậc hai, chúng ta có hai nghiệm khả dĩ
\[\xi_{\pm}=\frac{v^2}{gd}\pm\sqrt{\frac{v^4}{g^2d^2}-\frac{2v^2h}{gd^2}-1}.\]
Vì $\xi=\tan\alpha$ và $\tan$ là một hàm tăng (trong phạm vi của chúng ta), chúng ta quan tâm đến nghiệm nhỏ hơn. Do đó
\[\alpha=\arctan{\xi_-}.\]
Do đó, phương trình tầm xa cho chúng ta
\[x_{\max}=\frac{2\xi_-}{1+\xi_-^2}\frac{v^2}{g}.\]

Lập luận cho khoảng cách tối thiểu về cơ bản là giống nhau. Đỉnh của tường vẫn phải nằm dưới quỹ đạo và $\alpha>\ang{45}$ nên bây giờ chúng ta muốn $\alpha$ lớn nhất tương ứng với $\xi_+.$ Do đó
\[x_{\min}=\frac{2\xi_+}{1+\xi_+^2}\frac{v^2}{g}.\]
Và do đó
\[x_{\max}-x_{\min}=\frac{2v^2}{g}\left(\frac{\xi_-}{1+\xi_-^2}-\frac{\xi_+}{1+\xi_+^2}\right)\approx\boxed{5.38\;\mathrm{m}}.\]

Lưu ý để có một giải pháp đầy đủ (không cần thiết để có được câu trả lời số) người ta cũng nên kiểm tra xem vận tốc phóng đã cho có đủ lớn để vượt qua tường hay không. Vận tốc tối thiểu này tương đối dễ suy ra (đặc biệt là sử dụng đường bao):
\[v_{\min}=\sqrt{gh+g\sqrt{h^2+d^2}}\approx{12.6}\;\mathrm{m/s}<v,\]
vì vậy câu hỏi được đặt ra là hợp lý.
\end{solution}


\begin{solution}

Orient the sphere so its ends are vertical and align on the $z$ axis. The equipotential surfaces are rings with planes parallel to the $xy$ plane. The ring element at polar angle $\phi$ has length (along direction of current) $rd\phi$ and cross section area $2\pi r\sin\phi\cdot t$ (note that $r\gg t$). Integrating from angle $\phi_i=\frac{0.01}{30}$ to $\phi_f=\pi-\frac{0.01}{30}$, the total resistance of the filament is $$R=\int_{\phi_i}^{\phi_f}\frac{\rho\cdot rd\phi}{2\pi r\sin\phi\cdot t}=\boxed{277\;\Omega}.$$

% https://www.wolframalpha.com/input?i2d=true&i=Integrate%5Bcscx%2C%7Bx%2CDivide%5B0.01%2C30%5D%2Cpi-Divide%5B0.01%2C30%5D%7D%5D

\end{solution}
\begin{solution}

Orient the sphere so its ends are vertical and align on the $z$ axis. The equipotential surfaces are rings with planes parallel to the $xy$ plane. The ring element at polar angle $\phi$ has length (along direction of current) $rd\phi$ and cross section area $2\pi r\sin\phi\cdot t$ (note that $r\gg t$). Integrating from angle $\phi_i=\frac{0.01}{30}$ to $\phi_f=\pi-\frac{0.01}{30}$, the total resistance of the filament is $$R=\int_{\phi_i}^{\phi_f}\frac{\rho\cdot rd\phi}{2\pi r\sin\phi\cdot t}=\boxed{277\;\Omega}.$$

% https://www.wolframalpha.com/input?i2d=true&i=Integrate%5Bcscx%2C%7Bx%2CDivide%5B0.01%2C30%5D%2Cpi-Divide%5B0.01%2C30%5D%7D%5D

\end{solution}

%\begin{problem}{\textbf{\textsc{Filament 2}}\hspace{1mm}}
%Follin now modifies his filament to have six instead of two flattened circles arranged symmetrically around the sphere. He places contact wires at two flattened circles that are spaced $90^\circ$ apart on their shared great circle. Find the new resistance of the lightbulb.
%\end{problem}

\begin{problem}{\textbf{\textsc{Sợi đốt 2}}\hspace{1mm}}
	Follin giờ đây thay đổi sợi dây dẫn của mình để có sáu vòng tròn dẹt thay vì chỉ hai, được sắp xếp đối xứng xung quanh quả cầu. Ông đặt các dây tiếp xúc ở hai vòng tròn phẳng cách nhau $90^\circ$ trên đường tròn lớn chung của chúng. Tìm điện trở mới của bóng đèn.   
\end{problem}

%\begin{problem}{\textbf{\textsc{Filament 2}}\hspace{1mm}}
%Follin now modifies his filament to have six instead of two flattened circles arranged symmetrically around the sphere. He places contact wires at two flattened circles that are spaced $90^\circ$ apart on their shared great circle. Find the new resistance of the lightbulb.
%\end{problem}

\begin{problem}{\textbf{\textsc{Sợi đốt 2}}\hspace{1mm}}
	Follin giờ đây thay đổi sợi dây dẫn của mình để có sáu vòng tròn dẹt thay vì chỉ hai, được sắp xếp đối xứng xung quanh quả cầu. Ông đặt các dây tiếp xúc ở hai vòng tròn phẳng cách nhau $90^\circ$ trên đường tròn lớn chung của chúng. Tìm điện trở mới của bóng đèn.   
\end{problem}


\begin{solution}
The key idea is to realize that the 1D probability density function is analogous to the intensity from single slit diffraction. More precisely, let us consider the 1D probability density at a given position $x$ in the plane of the detector. $P = \norm{\Psi_x}^2 = \Psi_x ^{\ast} \Psi_x$ corresponds to taking the square of the norm of summed phasors, which is analogous to optical intensity being the square of the norm of the electric field phasor.
\newline
\newline
This makes our life easier, because we can simply reuse the derivations from classical wave optics! The DeBroglie wavelength of the neutrons is $$\lambda = \frac{h}{p} = \frac{h}{mv} = 601.1\;\mathrm{nm}$$ 

As can be derived using phasors (or simply by reusing the analogous formula from wave optics), we have
$$\norm{\Psi_x}^2 \propto {\left[\frac {\sin \left(\frac {\pi d \frac{x}{\sqrt{r^2 + x^2}}}{\lambda}\right)}{\left(\frac {\pi d \frac{x}{\sqrt{r^2 + x^2}}}{\lambda}\right)}\right]^2} $$
We have to be careful to normalize the probability density function when calculating the final result:
$$P = \frac {\int_{-w/2}^{w/2}\norm{\Psi_x}^2\,dx}{\int_{-\infty}^{\infty}\norm{\Psi_x}^2\,dx} $$
Plugging in the given parameters, this gives us a final percentage of $$P\times100\% = \boxed{90.2\%}$$


\end{solution}
\begin{solution}
The key idea is to realize that the 1D probability density function is analogous to the intensity from single slit diffraction. More precisely, let us consider the 1D probability density at a given position $x$ in the plane of the detector. $P = \norm{\Psi_x}^2 = \Psi_x ^{\ast} \Psi_x$ corresponds to taking the square of the norm of summed phasors, which is analogous to optical intensity being the square of the norm of the electric field phasor.
\newline
\newline
This makes our life easier, because we can simply reuse the derivations from classical wave optics! The DeBroglie wavelength of the neutrons is $$\lambda = \frac{h}{p} = \frac{h}{mv} = 601.1\;\mathrm{nm}$$ 

As can be derived using phasors (or simply by reusing the analogous formula from wave optics), we have
$$\norm{\Psi_x}^2 \propto {\left[\frac {\sin \left(\frac {\pi d \frac{x}{\sqrt{r^2 + x^2}}}{\lambda}\right)}{\left(\frac {\pi d \frac{x}{\sqrt{r^2 + x^2}}}{\lambda}\right)}\right]^2} $$
We have to be careful to normalize the probability density function when calculating the final result:
$$P = \frac {\int_{-w/2}^{w/2}\norm{\Psi_x}^2\,dx}{\int_{-\infty}^{\infty}\norm{\Psi_x}^2\,dx} $$
Plugging in the given parameters, this gives us a final percentage of $$P\times100\% = \boxed{90.2\%}$$


\end{solution}

%\begin{problem}{\textbf{\textsc{Peculiar Pump}}} A box with volume $V=1\;\mathrm{m}^3$ and initial temperature $T_0=100\;\mathrm{K}$ contains monatomic ideal gas initially kept at a constant low pressure $p_1=100\;\mathrm{Pa}$. The mass of the gas molecules is $m=7\times10^{-27}\;\mathrm{kg}$. The box is connected to a very large reservoir which contains monatomic ideal gas at temperature $T_2=400\;\mathrm{K}$ and pressure $p_2>p_1$. A small hole with area $A=10^{-8}\;\mathrm{m}^2$ is poked in the box at time $t=0\;\mathrm{days}$, allowing gas to escape. (The box is no longer at constant pressure after this point.) For every molecule of gas that escapes through this hole, 3 molecules of gas are let into the box from the reservoir. (The particles let in from the reservoir are selected with probability proportional to their velocity component perpendicular to the hole. In other words, the particles effuse through the hole but with the door to the hole restricting their number to 3 times the number of outgoing particles.) How many days does it take for the temperature in the box to double? Assume changes in the reservoir's pressure and temperature are negligible. For reference, the rate of particles escaping out of a hole with area A is given by:\begin{equation*}
		%\Phi=\frac{pA}{\sqrt{2\pi mk_BT}}
	%\end{equation*}
	%where $p$ is the pressure of the gas, $m$ is the mass of the gas molecules, $k_B$ is the Boltzmann constant and $T$ is the temperature of the gas. Please note that the average energy per particle of particles leaving via effusion is $2kT$.
	
%\end{problem}

\begin{problem}{\textbf{\textsc{Peculiar Pump}}} Một hộp có thể tích $V=1\;\mathrm{m}^3$ và nhiệt độ ban đầu $T_0=100\;\mathrm{K}$ chứa khí lý tưởng đơn nguyên tử được giữ ở áp suất thấp không đổi $p_1=100\;\mathrm{Pa}$. Khối lượng của phân tử khí là $m=7\times10^{-27}\;\mathrm{kg}$. Hộp được nối với một bể chứa rất lớn có chứa khí lý tưởng đơn nguyên tử ở nhiệt độ $T_2=400\;\mathrm{K}$ và áp suất $p_2>p_1$. Một lỗ nhỏ với diện tích $A=10^{-8}\;\mathrm{m}^2$ được khoan trên hộp vào thời điểm $t=0\;\mathrm{days}$, cho phép khí thoát ra. (Hộp không còn ở áp suất không đổi sau thời điểm này.) Với mỗi phân tử khí thoát ra qua lỗ này, có 3 phân tử khí từ khoang chứa được đưa vào hộp. (Các phân tử được đưa vào từ khoang chứa được chọn với xác suất tỷ lệ thuận với thành phần vận tốc của chúng vuông góc với lỗ. Nói cách khác, các phân tử khuếch tán qua lỗ nhưng cửa lỗ giới hạn số lượng của chúng là 3 lần số phân tử thoát ra ngoài.) Mất bao nhiêu ngày để nhiệt độ trong hộp tăng gấp đôi? Giả sử sự thay đổi áp suất và nhiệt độ trong khoang chứa là không đáng kể. Tham khảo, tốc độ phân tử thoát ra ngoài qua một lỗ có diện tích A được tính bởi:\begin{equation*}
    \Phi=\frac{pA}{\sqrt{2\pi mk_BT}}
\end{equation*}
trong đó $p$ là áp suất của khí, $m$ là khối lượng của phân tử khí, $k_B$ là hằng số Boltzmann và $T$ là nhiệt độ của khí. Xin lưu ý rằng năng lượng trung bình trên mỗi phân tử của các phân tử thoát ra qua quá trình khuếch tán là $2kT$.

\end{problem}
%\begin{problem}{\textbf{\textsc{Peculiar Pump}}} A box with volume $V=1\;\mathrm{m}^3$ and initial temperature $T_0=100\;\mathrm{K}$ contains monatomic ideal gas initially kept at a constant low pressure $p_1=100\;\mathrm{Pa}$. The mass of the gas molecules is $m=7\times10^{-27}\;\mathrm{kg}$. The box is connected to a very large reservoir which contains monatomic ideal gas at temperature $T_2=400\;\mathrm{K}$ and pressure $p_2>p_1$. A small hole with area $A=10^{-8}\;\mathrm{m}^2$ is poked in the box at time $t=0\;\mathrm{days}$, allowing gas to escape. (The box is no longer at constant pressure after this point.) For every molecule of gas that escapes through this hole, 3 molecules of gas are let into the box from the reservoir. (The particles let in from the reservoir are selected with probability proportional to their velocity component perpendicular to the hole. In other words, the particles effuse through the hole but with the door to the hole restricting their number to 3 times the number of outgoing particles.) How many days does it take for the temperature in the box to double? Assume changes in the reservoir's pressure and temperature are negligible. For reference, the rate of particles escaping out of a hole with area A is given by:\begin{equation*}
		%\Phi=\frac{pA}{\sqrt{2\pi mk_BT}}
	%\end{equation*}
	%where $p$ is the pressure of the gas, $m$ is the mass of the gas molecules, $k_B$ is the Boltzmann constant and $T$ is the temperature of the gas. Please note that the average energy per particle of particles leaving via effusion is $2kT$.
	
%\end{problem}

\begin{problem}{\textbf{\textsc{Peculiar Pump}}} Một hộp có thể tích $V=1\;\mathrm{m}^3$ và nhiệt độ ban đầu $T_0=100\;\mathrm{K}$ chứa khí lý tưởng đơn nguyên tử được giữ ở áp suất thấp không đổi $p_1=100\;\mathrm{Pa}$. Khối lượng của phân tử khí là $m=7\times10^{-27}\;\mathrm{kg}$. Hộp được nối với một bể chứa rất lớn có chứa khí lý tưởng đơn nguyên tử ở nhiệt độ $T_2=400\;\mathrm{K}$ và áp suất $p_2>p_1$. Một lỗ nhỏ với diện tích $A=10^{-8}\;\mathrm{m}^2$ được khoan trên hộp vào thời điểm $t=0\;\mathrm{days}$, cho phép khí thoát ra. (Hộp không còn ở áp suất không đổi sau thời điểm này.) Với mỗi phân tử khí thoát ra qua lỗ này, có 3 phân tử khí từ khoang chứa được đưa vào hộp. (Các phân tử được đưa vào từ khoang chứa được chọn với xác suất tỷ lệ thuận với thành phần vận tốc của chúng vuông góc với lỗ. Nói cách khác, các phân tử khuếch tán qua lỗ nhưng cửa lỗ giới hạn số lượng của chúng là 3 lần số phân tử thoát ra ngoài.) Mất bao nhiêu ngày để nhiệt độ trong hộp tăng gấp đôi? Giả sử sự thay đổi áp suất và nhiệt độ trong khoang chứa là không đáng kể. Tham khảo, tốc độ phân tử thoát ra ngoài qua một lỗ có diện tích A được tính bởi:\begin{equation*}
    \Phi=\frac{pA}{\sqrt{2\pi mk_BT}}
\end{equation*}
trong đó $p$ là áp suất của khí, $m$ là khối lượng của phân tử khí, $k_B$ là hằng số Boltzmann và $T$ là nhiệt độ của khí. Xin lưu ý rằng năng lượng trung bình trên mỗi phân tử của các phân tử thoát ra qua quá trình khuếch tán là $2kT$.

\end{problem}

\begin{problem}{\textbf{\textsc{Air Cushion}}}
An air cushion is in the shape of a cylinder with length $\ell=10.0\;\mathrm{m}$ and circular cross-sectional radius $R=28\;\mathrm{cm}$. The ends of the cylinder lie in vertical planes, and the length lies parallel to the horizontal ground. It is filled with an incompressible gas. Both the surface of the air cushion and gas inside have negligible weight compared to other forces in the scenario. The surface maintains a a constant surface tension $\gamma=5.0\;\mathrm{N/m}$ whenever deformed. A flat slab of mass $m=12.0\;\mathrm{kg}$ which is wider than the cushion is balanced on top, squishing the cushion. Find the new horizontal width of the cushion. Assume that its cross-section remains symmetric about a vertical axis.

\end{problem}
\begin{problem}{\textbf{\textsc{Air Cushion}}}
An air cushion is in the shape of a cylinder with length $\ell=10.0\;\mathrm{m}$ and circular cross-sectional radius $R=28\;\mathrm{cm}$. The ends of the cylinder lie in vertical planes, and the length lies parallel to the horizontal ground. It is filled with an incompressible gas. Both the surface of the air cushion and gas inside have negligible weight compared to other forces in the scenario. The surface maintains a a constant surface tension $\gamma=5.0\;\mathrm{N/m}$ whenever deformed. A flat slab of mass $m=12.0\;\mathrm{kg}$ which is wider than the cushion is balanced on top, squishing the cushion. Find the new horizontal width of the cushion. Assume that its cross-section remains symmetric about a vertical axis.

\end{problem}

%\begin{problem}
%{\textbf{\textsc{Soapy oscillator}}} On a smooth table lies a square frame made of four homogeneous rods of length $\ell=50\;\mathrm{cm}$ and mass $m=150\;\mathrm{g}$ which are hinged together at the corners. Each (massless) hinge carries a charge $q=1.5\cdot10^{-6}\;\mathrm{C}$. Before the frame was put on the table, it was put in a liquid soap mixture which left a soap film within the frame defined by the rods. The surface tension of said soapy water is $\sigma=0.035\;\mathrm{N/m}$. It turns out that it is possible to have small amplitude oscillations where opposing corners of the frame have opposite velocities (away/towards from the centre). What is the respective angular frequency of the oscillations?
%\end{problem}

\begin{problem}
	{\textbf{\textsc{Dao động xà phòng}}} Trên một mặt bàn trơn có một khung hình vuông làm bằng bốn thanh đồng nhất có chiều dài $\ell=50\;\mathrm{cm}$ và khối lượng $m=150\;\mathrm{g}$các thanh được gắn với nhau bằng bản lề ở các góc. Mỗi bản lề (không có khối lượng) mang điện tích $q=1.5\cdot10^{-6}\;\mathrm{C}$. Trước khi khung được đặt lên bàn, nó đã được nhúng vào hỗn hợp xà phòng lỏng để tạo thành một màng xà phòng bên trong khung được định hình bởi các thanh. Sức căng bề mặt của nước xà phòng là $\sigma=0.035\;\mathrm{N/m}$. Kết quả là có thể có dao động với biên độ nhỏ mà tại đó các góc đối diện của khung có vận tốc ngược chiều nhau (hướng ra xa hoặc về phía trung tâm). Tần số góc của dao động này là bao nhiêu?
\end{problem}

%\begin{problem}
%{\textbf{\textsc{Soapy oscillator}}} On a smooth table lies a square frame made of four homogeneous rods of length $\ell=50\;\mathrm{cm}$ and mass $m=150\;\mathrm{g}$ which are hinged together at the corners. Each (massless) hinge carries a charge $q=1.5\cdot10^{-6}\;\mathrm{C}$. Before the frame was put on the table, it was put in a liquid soap mixture which left a soap film within the frame defined by the rods. The surface tension of said soapy water is $\sigma=0.035\;\mathrm{N/m}$. It turns out that it is possible to have small amplitude oscillations where opposing corners of the frame have opposite velocities (away/towards from the centre). What is the respective angular frequency of the oscillations?
%\end{problem}

\begin{problem}
	{\textbf{\textsc{Dao động xà phòng}}} Trên một mặt bàn trơn có một khung hình vuông làm bằng bốn thanh đồng nhất có chiều dài $\ell=50\;\mathrm{cm}$ và khối lượng $m=150\;\mathrm{g}$các thanh được gắn với nhau bằng bản lề ở các góc. Mỗi bản lề (không có khối lượng) mang điện tích $q=1.5\cdot10^{-6}\;\mathrm{C}$. Trước khi khung được đặt lên bàn, nó đã được nhúng vào hỗn hợp xà phòng lỏng để tạo thành một màng xà phòng bên trong khung được định hình bởi các thanh. Sức căng bề mặt của nước xà phòng là $\sigma=0.035\;\mathrm{N/m}$. Kết quả là có thể có dao động với biên độ nhỏ mà tại đó các góc đối diện của khung có vận tốc ngược chiều nhau (hướng ra xa hoặc về phía trung tâm). Tần số góc của dao động này là bao nhiêu?
\end{problem}




% \begin{solution}

% We first transform into the reference frame of the particle. In this frame, it becomes a stationary particle receiving and emitting light isotropically. What frequency of light, we ask? We will perform a Lorentz transformation to find out. Let the x-axis be along the direction of the particle velocity. Setting $c=1$, the wavevector 4-vector in the lab frame is 

% $$k^{\mu}=(\omega, \omega  \cos \alpha, -\omega \sin \alpha ) $$

% Transforming to the particle frame, the new 4-vector is 

% $$k_1^{\mu}=(\omega \gamma (1-v\cos\alpha), ..., ...)$$

% The x- and y- components do not matter as we only care about the t-component, which tells us the frequency of the light in the particle frame. So light of frequency $\omega_1=\omega \gamma (1-v\cos\alpha)$ gets scattered into all directions. Consider light that gets scattered into an arbitrary angle $\phi$ from the x-axis. Its 4-vector would be

% $$k_2^{\mu}=(\omega_1, \omega_1 \cos \phi, \omega_1 \sin \phi)$$

% Transforming this back into the lab frame, the final 4-vector is 

% $$k_3^{\mu}=(\omega_1 \gamma (1+v\cos\phi), \omega_1 \gamma (\cos\phi+v), \omega_1 \sin \phi)$$

% In the lab frame, the angle $\delta$ between the x-axis and the direction of propagation of $k_3$ is given by

% $$\tan \delta = \frac{\sin \phi}{\gamma(\cos\phi+v)}$$

% We can start plugging in numbers. We find that $\gamma=1.155$, $\omega_1=5.29\cdot 10^{15} \ \mathrm{Hz}$, and $\tan \delta=0.966$. We can solve for $\phi$, getting back $\phi=70.0^{\circ}$. 

% Our answer is therefore $\boxed{\omega'=k_3^0=7.15\cdot 10^{15} \ \mathrm{Hz}}$.

% \end{solution}

\begin{solution}

Đầu tiên, chúng ta chuyển đổi vào hệ quy chiếu của hạt. Trong hệ này, nó trở thành một hạt đứng yên nhận và phát ra ánh sáng đẳng hướng. Chúng ta hỏi tần số của ánh sáng là bao nhiêu? Chúng ta sẽ thực hiện một phép biến đổi Lorentz để tìm ra. Đặt trục x dọc theo hướng vận tốc của hạt. Đặt $c=1$, vectơ sóng 4 chiều trong hệ phòng thí nghiệm là 

$$k^{\mu}=(\omega, \omega  \cos \alpha, -\omega \sin \alpha ) $$

Chuyển đổi sang hệ quy chiếu của hạt, vectơ 4 chiều mới là 

$$k_1^{\mu}=(\omega \gamma (1-v\cos\alpha), ..., ...)$$

Các thành phần x và y không quan trọng vì chúng ta chỉ quan tâm đến thành phần t, thành phần này cho chúng ta biết tần số của ánh sáng trong hệ quy chiếu của hạt. Vì vậy, ánh sáng có tần số $\omega_1=\omega \gamma (1-v\cos\alpha)$ bị tán xạ theo mọi hướng. Xem xét ánh sáng bị tán xạ theo một góc bất kỳ $\phi$ từ trục x. Vectơ 4 chiều của nó sẽ là

$$k_2^{\mu}=(\omega_1, \omega_1 \cos \phi, \omega_1 \sin \phi)$$

Chuyển đổi điều này trở lại hệ phòng thí nghiệm, vectơ 4 chiều cuối cùng là 

$$k_3^{\mu}=(\omega_1 \gamma (1+v\cos\phi), \omega_1 \gamma (\cos\phi+v), \omega_1 \sin \phi)$$

Trong hệ phòng thí nghiệm, góc $\delta$ giữa trục x và hướng truyền của $k_3$ được cho bởi

$$\tan \delta = \frac{\sin \phi}{\gamma(\cos\phi+v)}$$

Chúng ta có thể bắt đầu thay số vào. Chúng ta tìm thấy $\gamma=1.155$, $\omega_1=5.29\cdot 10^{15} \ \mathrm{Hz}$, và $\tan \delta=0.966$. Chúng ta có thể giải $\phi$, nhận lại $\phi=70.0^{\circ}$. 

Do đó, câu trả lời của chúng ta là $\boxed{\omega'=k_3^0=7.15\cdot 10^{15} \ \mathrm{Hz}}$.

\end{solution}



% \begin{solution}

% We first transform into the reference frame of the particle. In this frame, it becomes a stationary particle receiving and emitting light isotropically. What frequency of light, we ask? We will perform a Lorentz transformation to find out. Let the x-axis be along the direction of the particle velocity. Setting $c=1$, the wavevector 4-vector in the lab frame is 

% $$k^{\mu}=(\omega, \omega  \cos \alpha, -\omega \sin \alpha ) $$

% Transforming to the particle frame, the new 4-vector is 

% $$k_1^{\mu}=(\omega \gamma (1-v\cos\alpha), ..., ...)$$

% The x- and y- components do not matter as we only care about the t-component, which tells us the frequency of the light in the particle frame. So light of frequency $\omega_1=\omega \gamma (1-v\cos\alpha)$ gets scattered into all directions. Consider light that gets scattered into an arbitrary angle $\phi$ from the x-axis. Its 4-vector would be

% $$k_2^{\mu}=(\omega_1, \omega_1 \cos \phi, \omega_1 \sin \phi)$$

% Transforming this back into the lab frame, the final 4-vector is 

% $$k_3^{\mu}=(\omega_1 \gamma (1+v\cos\phi), \omega_1 \gamma (\cos\phi+v), \omega_1 \sin \phi)$$

% In the lab frame, the angle $\delta$ between the x-axis and the direction of propagation of $k_3$ is given by

% $$\tan \delta = \frac{\sin \phi}{\gamma(\cos\phi+v)}$$

% We can start plugging in numbers. We find that $\gamma=1.155$, $\omega_1=5.29\cdot 10^{15} \ \mathrm{Hz}$, and $\tan \delta=0.966$. We can solve for $\phi$, getting back $\phi=70.0^{\circ}$. 

% Our answer is therefore $\boxed{\omega'=k_3^0=7.15\cdot 10^{15} \ \mathrm{Hz}}$.

% \end{solution}

\begin{solution}

Đầu tiên, chúng ta chuyển đổi vào hệ quy chiếu của hạt. Trong hệ này, nó trở thành một hạt đứng yên nhận và phát ra ánh sáng đẳng hướng. Chúng ta hỏi tần số của ánh sáng là bao nhiêu? Chúng ta sẽ thực hiện một phép biến đổi Lorentz để tìm ra. Đặt trục x dọc theo hướng vận tốc của hạt. Đặt $c=1$, vectơ sóng 4 chiều trong hệ phòng thí nghiệm là 

$$k^{\mu}=(\omega, \omega  \cos \alpha, -\omega \sin \alpha ) $$

Chuyển đổi sang hệ quy chiếu của hạt, vectơ 4 chiều mới là 

$$k_1^{\mu}=(\omega \gamma (1-v\cos\alpha), ..., ...)$$

Các thành phần x và y không quan trọng vì chúng ta chỉ quan tâm đến thành phần t, thành phần này cho chúng ta biết tần số của ánh sáng trong hệ quy chiếu của hạt. Vì vậy, ánh sáng có tần số $\omega_1=\omega \gamma (1-v\cos\alpha)$ bị tán xạ theo mọi hướng. Xem xét ánh sáng bị tán xạ theo một góc bất kỳ $\phi$ từ trục x. Vectơ 4 chiều của nó sẽ là

$$k_2^{\mu}=(\omega_1, \omega_1 \cos \phi, \omega_1 \sin \phi)$$

Chuyển đổi điều này trở lại hệ phòng thí nghiệm, vectơ 4 chiều cuối cùng là 

$$k_3^{\mu}=(\omega_1 \gamma (1+v\cos\phi), \omega_1 \gamma (\cos\phi+v), \omega_1 \sin \phi)$$

Trong hệ phòng thí nghiệm, góc $\delta$ giữa trục x và hướng truyền của $k_3$ được cho bởi

$$\tan \delta = \frac{\sin \phi}{\gamma(\cos\phi+v)}$$

Chúng ta có thể bắt đầu thay số vào. Chúng ta tìm thấy $\gamma=1.155$, $\omega_1=5.29\cdot 10^{15} \ \mathrm{Hz}$, và $\tan \delta=0.966$. Chúng ta có thể giải $\phi$, nhận lại $\phi=70.0^{\circ}$. 

Do đó, câu trả lời của chúng ta là $\boxed{\omega'=k_3^0=7.15\cdot 10^{15} \ \mathrm{Hz}}$.

\end{solution}


%\begin{problem}{\textsc{\textbf{Reluctant Roller}}}
%A hoop of mass $m$ and radius $r$ rests on a surface with coefficient of friction $\mu$. At time $t=0$, a string is attached to the hoop's highest point and a constant horizontal tension $T$ is applied. By time $t$, the hoop has rotated by angle $\theta(t)$. What is the minimum value of $\displaystyle\frac{T}{\mu mg}$ such that $\theta(t)$ has a local maximum (i.e., is not strictly increasing)? You may need to graph an implicit function.
%\vspace{-0.5cm}
%\begin{center}
%\begin{tikzpicture}[dot/.style = {circle, fill, minimum size=#1, inner %sep=0pt, outer sep=0pt}]
%\draw[very thick] (0,0) circle (1.5);
%\draw (0,0) -- node[above]{$r$} (-1.4,0);
%\node at (0.9, 1.2) [dot=5]{};
%\draw[->] (0.9, 1.2) -- (5, 1.2) node[right] {$T$};
%\draw[very thick] (-2.5, -1.5) -- (6, -1.5);
%\draw (0, 1.7) -- (0, 2);
%\draw (0, 1.85) arc (90 : 53 : 1.85) node[midway, above]{$\theta$};
%\draw (1.02, 1.36) -- (1.2, 1.6);
%\end{tikzpicture}
%\end{center}
%\end{problem}

\begin{problem}{\textsc{\textbf{Con lăn cưỡng bức}}}
	Một vòng tròn có khối lượng $m$ và bán kính $r$ nằm trên một bề mặt có hệ số ma sát  $\mu$. Tại thời điểm $t=0$, một sợi dây được gắn vào điểm cao nhất của vòng tròn và một lực căng ngang không đổi $T$ được kéo. Đến thời điểm $t$, vòng tròn đã quay một góc $\theta(t)$. Tìm giá trị nhỏ nhất của $\displaystyle\frac{T}{\mu mg}$ sao cho $\theta(t)$ có một cực đại cục bộ (tức là không tăng đều). Có thể cần phải vẽ đồ thị của một hàm ẩn.
	\vspace{-0.5cm}
	\begin{center}
		\begin{tikzpicture}[dot/.style = {circle, fill, minimum size=#1, inner sep=0pt, outer sep=0pt}]
			\draw[very thick] (0,0) circle (1.5);
			\draw (0,0) -- node[above]{$r$} (-1.4,0);
			\node at (0.9, 1.2) [dot=5]{};
			\draw[->] (0.9, 1.2) -- (5, 1.2) node[right] {$T$};
			\draw[very thick] (-2.5, -1.5) -- (6, -1.5);
			\draw (0, 1.7) -- (0, 2);
			\draw (0, 1.85) arc (90 : 53 : 1.85) node[midway, above]{$\theta$};
			\draw (1.02, 1.36) -- (1.2, 1.6);
		\end{tikzpicture}
	\end{center}
\end{problem}

%\begin{problem}{\textsc{\textbf{Reluctant Roller}}}
%A hoop of mass $m$ and radius $r$ rests on a surface with coefficient of friction $\mu$. At time $t=0$, a string is attached to the hoop's highest point and a constant horizontal tension $T$ is applied. By time $t$, the hoop has rotated by angle $\theta(t)$. What is the minimum value of $\displaystyle\frac{T}{\mu mg}$ such that $\theta(t)$ has a local maximum (i.e., is not strictly increasing)? You may need to graph an implicit function.
%\vspace{-0.5cm}
%\begin{center}
%\begin{tikzpicture}[dot/.style = {circle, fill, minimum size=#1, inner %sep=0pt, outer sep=0pt}]
%\draw[very thick] (0,0) circle (1.5);
%\draw (0,0) -- node[above]{$r$} (-1.4,0);
%\node at (0.9, 1.2) [dot=5]{};
%\draw[->] (0.9, 1.2) -- (5, 1.2) node[right] {$T$};
%\draw[very thick] (-2.5, -1.5) -- (6, -1.5);
%\draw (0, 1.7) -- (0, 2);
%\draw (0, 1.85) arc (90 : 53 : 1.85) node[midway, above]{$\theta$};
%\draw (1.02, 1.36) -- (1.2, 1.6);
%\end{tikzpicture}
%\end{center}
%\end{problem}

\begin{problem}{\textsc{\textbf{Con lăn cưỡng bức}}}
	Một vòng tròn có khối lượng $m$ và bán kính $r$ nằm trên một bề mặt có hệ số ma sát  $\mu$. Tại thời điểm $t=0$, một sợi dây được gắn vào điểm cao nhất của vòng tròn và một lực căng ngang không đổi $T$ được kéo. Đến thời điểm $t$, vòng tròn đã quay một góc $\theta(t)$. Tìm giá trị nhỏ nhất của $\displaystyle\frac{T}{\mu mg}$ sao cho $\theta(t)$ có một cực đại cục bộ (tức là không tăng đều). Có thể cần phải vẽ đồ thị của một hàm ẩn.
	\vspace{-0.5cm}
	\begin{center}
		\begin{tikzpicture}[dot/.style = {circle, fill, minimum size=#1, inner sep=0pt, outer sep=0pt}]
			\draw[very thick] (0,0) circle (1.5);
			\draw (0,0) -- node[above]{$r$} (-1.4,0);
			\node at (0.9, 1.2) [dot=5]{};
			\draw[->] (0.9, 1.2) -- (5, 1.2) node[right] {$T$};
			\draw[very thick] (-2.5, -1.5) -- (6, -1.5);
			\draw (0, 1.7) -- (0, 2);
			\draw (0, 1.85) arc (90 : 53 : 1.85) node[midway, above]{$\theta$};
			\draw (1.02, 1.36) -- (1.2, 1.6);
		\end{tikzpicture}
	\end{center}
\end{problem}


\begin{solution}

Since the resistance is small, the flux $\Phi=LI$ remains constant. The given inductance can be written as $L_0=\frac{\mu_0 N^2 \pi r^2}{\ell_0}$, where $N$ is the number of turns and $r$ is the radius. Displace the ring by a small distance $x$. The inductance of the solenoid at this moment is $$L=\frac{\mu_0 N^2\pi r^2}{\ell_0+x}=\frac{L_0}{1+x/\ell_0},$$ so $$I=I_0(1+\frac{x}{\ell_0}).$$

The energy of the system is $E=\frac12 LI^2 + \frac12 kx^2 + \frac12 m\dot{x}^2$. Since the power loss through the resistor is small, we have
\begin{align*}
    0=\frac{dE}{dt} &= L_0I_0\dot{I} + kx\dot{x} + m\dot{x}\ddot{x} \\
    0 &= L_0I_0^2/\ell_0 + kx + m\ddot{x}.
\end{align*}

From here we read off angular frequency $\omega=\sqrt{\frac{k}{m}}=\boxed{14.1\;\mathrm{rad/s}}.$

\end{solution}
\begin{solution}

Since the resistance is small, the flux $\Phi=LI$ remains constant. The given inductance can be written as $L_0=\frac{\mu_0 N^2 \pi r^2}{\ell_0}$, where $N$ is the number of turns and $r$ is the radius. Displace the ring by a small distance $x$. The inductance of the solenoid at this moment is $$L=\frac{\mu_0 N^2\pi r^2}{\ell_0+x}=\frac{L_0}{1+x/\ell_0},$$ so $$I=I_0(1+\frac{x}{\ell_0}).$$

The energy of the system is $E=\frac12 LI^2 + \frac12 kx^2 + \frac12 m\dot{x}^2$. Since the power loss through the resistor is small, we have
\begin{align*}
    0=\frac{dE}{dt} &= L_0I_0\dot{I} + kx\dot{x} + m\dot{x}\ddot{x} \\
    0 &= L_0I_0^2/\ell_0 + kx + m\ddot{x}.
\end{align*}

From here we read off angular frequency $\omega=\sqrt{\frac{k}{m}}=\boxed{14.1\;\mathrm{rad/s}}.$

\end{solution}

%%Thanks to Joshua Wang (USPT '24) for providing the non-fakesolve solution! :)
\begin{solution}
Let the string have tension $T$, and let the displacement be $A\sin(\pi x/L)\sin(\omega t)$. We compute the total energy by finding the length at the peak of an oscillation:
\begin{align*}E&=T(L_{\text{peak}} - L)\\ &=T\left(\int_0^L\sqrt{1 + \frac{\pi^2 A^2}{L^2}\cos^2\frac{\pi x}{L}}\ dx - L\right)\\ &\approx T\left(\int_0^L 1 + \frac{\pi^2 A^2}{2L^2}\cos^2\frac{\pi x}{L}\ dx - L\right)\\ &= \frac{\pi^2 A^2 T}{4L}\end{align*}
At time $t$, the net horizontal force on the finger is:
\begin{align*}F_x = T\left(1 - \cos\left(\arctan\left(\frac{A\pi}{L}\sin(\omega t)\right)\right)\right)\approx \frac{\pi^2 A^2 T}{2L^2}\sin^2(\omega t)\end{align*}
Thus, the average horizontal force on the finger is $\displaystyle \langle F_x\rangle = -\frac{dE}{dL} = \frac{\pi^2 A^2 T}{4L^2} = \frac{E}{L}$. This gives $\displaystyle E \propto \frac{1}{L}$, so $A$ is constant, and $A_2=A=\boxed{1.25\times 10^{-3}\;\mathrm{m}}.$
\newline
\newline
Note that the question could also have been solved through the adiabatic invariant, or alternately by realizing that the restoring force is linear in displacement at all points on the string. As a result, a valid solution, by superposition, is to use a rotational analogy (with the string performing rotation like a skipping-rope) and conserve angular momentum. In that case, it will be observed that the decrease in mass of the vibrating portion of the string exactly counteracts the increase in angular velocity caused by the decreased length.  

\end{solution}
%%Thanks to Joshua Wang (USPT '24) for providing the non-fakesolve solution! :)
\begin{solution}
Let the string have tension $T$, and let the displacement be $A\sin(\pi x/L)\sin(\omega t)$. We compute the total energy by finding the length at the peak of an oscillation:
\begin{align*}E&=T(L_{\text{peak}} - L)\\ &=T\left(\int_0^L\sqrt{1 + \frac{\pi^2 A^2}{L^2}\cos^2\frac{\pi x}{L}}\ dx - L\right)\\ &\approx T\left(\int_0^L 1 + \frac{\pi^2 A^2}{2L^2}\cos^2\frac{\pi x}{L}\ dx - L\right)\\ &= \frac{\pi^2 A^2 T}{4L}\end{align*}
At time $t$, the net horizontal force on the finger is:
\begin{align*}F_x = T\left(1 - \cos\left(\arctan\left(\frac{A\pi}{L}\sin(\omega t)\right)\right)\right)\approx \frac{\pi^2 A^2 T}{2L^2}\sin^2(\omega t)\end{align*}
Thus, the average horizontal force on the finger is $\displaystyle \langle F_x\rangle = -\frac{dE}{dL} = \frac{\pi^2 A^2 T}{4L^2} = \frac{E}{L}$. This gives $\displaystyle E \propto \frac{1}{L}$, so $A$ is constant, and $A_2=A=\boxed{1.25\times 10^{-3}\;\mathrm{m}}.$
\newline
\newline
Note that the question could also have been solved through the adiabatic invariant, or alternately by realizing that the restoring force is linear in displacement at all points on the string. As a result, a valid solution, by superposition, is to use a rotational analogy (with the string performing rotation like a skipping-rope) and conserve angular momentum. In that case, it will be observed that the decrease in mass of the vibrating portion of the string exactly counteracts the increase in angular velocity caused by the decreased length.  

\end{solution}

% \newpage
% \begin{solution}
% If we reach a steady state, we can assign a probability $p(x)$ to each cell $x$.
% We can think of probability jumping, instead of the ball jumping.
% So at one step, all the probability in $x$ disappears, and we need some probability jump back into it.\\

% The probability jumping into $x$ is:
% \begin{equation*}
%     p(x) = \sum_{u \in \text{neighbours(x)}} p(u)p(u \rightarrow x)   
% \end{equation*}
% Now $p(u\rightarrow x)$ is just $\frac{1}{\text{deg}(u)}$. The degree of a cell, denoted $\text{deg}$ is the number
% of other cells that can reach it
% \begin{equation*}
%     p(x) = \sum_{u \in \text{neighbours(x)}} \frac{p(u)}{\text{deg}(u)}  
% \end{equation*}

% Here, we can see that if the system reaches a steady state, $p(x)$ will be proportional to $\text{deg}(x)$.
% So the solution is 
% \begin{equation*}
%     p(x) = \frac{\text{deg}(x)}{\sum_u \text{deg}(u)}
% \end{equation*}

% Which we can check in practice (with something like Mathematica), and it works.
% \\

% For our $4 \times 4$ grid, the degrees of the cells are:

% \[
% \begin{matrix}
% 2 & 3 & 3 & 2 \\
% 3 & 4 & 4 & 3 \\
% 3 & 4 & 4 & 3 \\
% 2 & 3 & 3 & 2
% \end{matrix}
% \]

% The probabilities will be proportional to these.\\
% %The total sum is $52$, and because one particle starts in each cell, we can just multiply it by $16$. This gives 
% %\[
% %\frac{x \cdot 16}{52} = \frac{x \cdot 4}{13}
% %\]
% %for each \( x \) in the second row. So the expected number of particles in each cell is
% %\[
% %\frac{12}{13}, \quad \frac{20}{13}, \quad \frac{20}{13}, \quad \frac{12}{13},
% %\]
% % and $A\cdot B\cdot C\cdot D=\left(\frac{12}{13}\cdot\frac{20}{13}\right)^2=\frac{240}{169}\implies \boxed{409}.$

% Now let's find the energies. Denote the energy of cell $i$ as $E_i$. Also, we will denote $\beta = \frac{1}{k_B T}$. 
% %The probability of begin in that cell is

% %\[
% %\frac{e^{-\beta E_i}}{\sum_j{e^{-\beta E_j}}}
% %\]

% %This means that the ratio of probabilities of two states is:
% By the Boltzmann distribution:
% \[
% \frac{p(E_1)}{p(E_2)} = e^{\beta(E_2 - E_1)}
% \]
% \[
% E_2 - E_1 = k_B T \log{\frac{p(E_1)}{p(E_2)}}
% \]

% This means that the difference in energies is maximized for biggest ratio. Which gives:

% \[
% \Delta E_{max} = k_B T \log{\frac{4}{2}} = \boxed{5.9728 \cdot 10 ^ {-8}\mathrm{eV}.}
% \]



% \end{solution}

\begin{solution}
Ở trạng thái ổn định, chúng ta có thể gán một xác suất $p(x)$ cho mỗi ô $x$.
Chúng ta có thể nghĩ về xác suất nhảy, thay vì quả bóng nhảy.
Vì vậy, tại một bước, tất cả xác suất trong $x$ biến mất, và chúng ta cần một số xác suất nhảy trở lại vào nó.\\

Xác suất nhảy vào $x$ là:
\begin{equation*}
    p(x) = \sum_{u \in \text{neighbours(x)}} p(u)p(u \rightarrow x)   
\end{equation*}
Bây giờ $p(u\rightarrow x)$ chỉ là $\frac{1}{\text{deg}(u)}$. Bậc của một ô, ký hiệu là $\text{deg}$ là số
các ô khác có thể đến được nó
\begin{equation*}
    p(x) = \sum_{u \in \text{neighbours(x)}} \frac{p(u)}{\text{deg}(u)}  
\end{equation*}

Ở đây, chúng ta có thể thấy rằng nếu hệ thống đạt đến trạng thái ổn định, $p(x)$ sẽ tỷ lệ thuận với $\text{deg}(x)$.
Vì vậy, giải pháp là 
\begin{equation*}
    p(x) = \frac{\text{deg}(x)}{\sum_u \text{deg}(u)}
\end{equation*}

Điều này chúng ta có thể kiểm tra trong thực tế (với một cái gì đó như Mathematica), và nó hoạt động.
\\

Đối với lưới $4 \times 4$ của chúng ta, bậc của các ô là:

\[
\begin{matrix}
2 & 3 & 3 & 2 \\
3 & 4 & 4 & 3 \\
3 & 4 & 4 & 3 \\
2 & 3 & 3 & 2
\end{matrix}
\]

Xác suất sẽ tỷ lệ thuận với những điều này.\\
%Tổng số là $52$, và vì một hạt bắt đầu ở mỗi ô, chúng ta có thể nhân nó với $16$. Điều này cho 
%\[
%\frac{x \cdot 16}{52} = \frac{x \cdot 4}{13}
%\]
%cho mỗi \( x \) trong hàng thứ hai. Vì vậy, số hạt dự kiến trong mỗi ô là
%\[
%\frac{12}{13}, \quad \frac{20}{13}, \quad \frac{20}{13}, \quad \frac{12}{13},
%\]
% và $A\cdot B\cdot C\cdot D=\left(\frac{12}{13}\cdot\frac{20}{13}\right)^2=\frac{240}{169}\implies \boxed{409}.$

Bây giờ hãy tìm các năng lượng. Ký hiệu năng lượng của ô $i$ là $E_i$. Ngoài ra, chúng ta sẽ ký hiệu $\beta = \frac{1}{k_B T}$. 
%Xác suất bắt đầu trong ô đó là

%\[
%\frac{e^{-\beta E_i}}{\sum_j{e^{-\beta E_j}}}
%\]

%Điều này có nghĩa là tỷ lệ xác suất của hai trạng thái là:
Theo phân bố Boltzmann:
\[
\frac{p(E_1)}{p(E_2)} = e^{\beta(E_2 - E_1)}
\]
\[
E_2 - E_1 = k_B T \log{\frac{p(E_1)}{p(E_2)}}
\]

Điều này có nghĩa là sự chênh lệch năng lượng được tối đa hóa cho tỷ lệ lớn nhất. Điều này cho:

\[
\Delta E_{max} = k_B T \log{\frac{4}{2}} = \boxed{5.9728 \cdot 10 ^ {-8}\mathrm{eV}.}
\]



\end{solution}

% \newpage
% \begin{solution}
% If we reach a steady state, we can assign a probability $p(x)$ to each cell $x$.
% We can think of probability jumping, instead of the ball jumping.
% So at one step, all the probability in $x$ disappears, and we need some probability jump back into it.\\

% The probability jumping into $x$ is:
% \begin{equation*}
%     p(x) = \sum_{u \in \text{neighbours(x)}} p(u)p(u \rightarrow x)   
% \end{equation*}
% Now $p(u\rightarrow x)$ is just $\frac{1}{\text{deg}(u)}$. The degree of a cell, denoted $\text{deg}$ is the number
% of other cells that can reach it
% \begin{equation*}
%     p(x) = \sum_{u \in \text{neighbours(x)}} \frac{p(u)}{\text{deg}(u)}  
% \end{equation*}

% Here, we can see that if the system reaches a steady state, $p(x)$ will be proportional to $\text{deg}(x)$.
% So the solution is 
% \begin{equation*}
%     p(x) = \frac{\text{deg}(x)}{\sum_u \text{deg}(u)}
% \end{equation*}

% Which we can check in practice (with something like Mathematica), and it works.
% \\

% For our $4 \times 4$ grid, the degrees of the cells are:

% \[
% \begin{matrix}
% 2 & 3 & 3 & 2 \\
% 3 & 4 & 4 & 3 \\
% 3 & 4 & 4 & 3 \\
% 2 & 3 & 3 & 2
% \end{matrix}
% \]

% The probabilities will be proportional to these.\\
% %The total sum is $52$, and because one particle starts in each cell, we can just multiply it by $16$. This gives 
% %\[
% %\frac{x \cdot 16}{52} = \frac{x \cdot 4}{13}
% %\]
% %for each \( x \) in the second row. So the expected number of particles in each cell is
% %\[
% %\frac{12}{13}, \quad \frac{20}{13}, \quad \frac{20}{13}, \quad \frac{12}{13},
% %\]
% % and $A\cdot B\cdot C\cdot D=\left(\frac{12}{13}\cdot\frac{20}{13}\right)^2=\frac{240}{169}\implies \boxed{409}.$

% Now let's find the energies. Denote the energy of cell $i$ as $E_i$. Also, we will denote $\beta = \frac{1}{k_B T}$. 
% %The probability of begin in that cell is

% %\[
% %\frac{e^{-\beta E_i}}{\sum_j{e^{-\beta E_j}}}
% %\]

% %This means that the ratio of probabilities of two states is:
% By the Boltzmann distribution:
% \[
% \frac{p(E_1)}{p(E_2)} = e^{\beta(E_2 - E_1)}
% \]
% \[
% E_2 - E_1 = k_B T \log{\frac{p(E_1)}{p(E_2)}}
% \]

% This means that the difference in energies is maximized for biggest ratio. Which gives:

% \[
% \Delta E_{max} = k_B T \log{\frac{4}{2}} = \boxed{5.9728 \cdot 10 ^ {-8}\mathrm{eV}.}
% \]



% \end{solution}

\begin{solution}
Ở trạng thái ổn định, chúng ta có thể gán một xác suất $p(x)$ cho mỗi ô $x$.
Chúng ta có thể nghĩ về xác suất nhảy, thay vì quả bóng nhảy.
Vì vậy, tại một bước, tất cả xác suất trong $x$ biến mất, và chúng ta cần một số xác suất nhảy trở lại vào nó.\\

Xác suất nhảy vào $x$ là:
\begin{equation*}
    p(x) = \sum_{u \in \text{neighbours(x)}} p(u)p(u \rightarrow x)   
\end{equation*}
Bây giờ $p(u\rightarrow x)$ chỉ là $\frac{1}{\text{deg}(u)}$. Bậc của một ô, ký hiệu là $\text{deg}$ là số
các ô khác có thể đến được nó
\begin{equation*}
    p(x) = \sum_{u \in \text{neighbours(x)}} \frac{p(u)}{\text{deg}(u)}  
\end{equation*}

Ở đây, chúng ta có thể thấy rằng nếu hệ thống đạt đến trạng thái ổn định, $p(x)$ sẽ tỷ lệ thuận với $\text{deg}(x)$.
Vì vậy, giải pháp là 
\begin{equation*}
    p(x) = \frac{\text{deg}(x)}{\sum_u \text{deg}(u)}
\end{equation*}

Điều này chúng ta có thể kiểm tra trong thực tế (với một cái gì đó như Mathematica), và nó hoạt động.
\\

Đối với lưới $4 \times 4$ của chúng ta, bậc của các ô là:

\[
\begin{matrix}
2 & 3 & 3 & 2 \\
3 & 4 & 4 & 3 \\
3 & 4 & 4 & 3 \\
2 & 3 & 3 & 2
\end{matrix}
\]

Xác suất sẽ tỷ lệ thuận với những điều này.\\
%Tổng số là $52$, và vì một hạt bắt đầu ở mỗi ô, chúng ta có thể nhân nó với $16$. Điều này cho 
%\[
%\frac{x \cdot 16}{52} = \frac{x \cdot 4}{13}
%\]
%cho mỗi \( x \) trong hàng thứ hai. Vì vậy, số hạt dự kiến trong mỗi ô là
%\[
%\frac{12}{13}, \quad \frac{20}{13}, \quad \frac{20}{13}, \quad \frac{12}{13},
%\]
% và $A\cdot B\cdot C\cdot D=\left(\frac{12}{13}\cdot\frac{20}{13}\right)^2=\frac{240}{169}\implies \boxed{409}.$

Bây giờ hãy tìm các năng lượng. Ký hiệu năng lượng của ô $i$ là $E_i$. Ngoài ra, chúng ta sẽ ký hiệu $\beta = \frac{1}{k_B T}$. 
%Xác suất bắt đầu trong ô đó là

%\[
%\frac{e^{-\beta E_i}}{\sum_j{e^{-\beta E_j}}}
%\]

%Điều này có nghĩa là tỷ lệ xác suất của hai trạng thái là:
Theo phân bố Boltzmann:
\[
\frac{p(E_1)}{p(E_2)} = e^{\beta(E_2 - E_1)}
\]
\[
E_2 - E_1 = k_B T \log{\frac{p(E_1)}{p(E_2)}}
\]

Điều này có nghĩa là sự chênh lệch năng lượng được tối đa hóa cho tỷ lệ lớn nhất. Điều này cho:

\[
\Delta E_{max} = k_B T \log{\frac{4}{2}} = \boxed{5.9728 \cdot 10 ^ {-8}\mathrm{eV}.}
\]



\end{solution}


\begin{problem}{\textbf{\textsc{Bản dẫn}}} \textbf{[Bài toán này đã bị xóa khỏi cuộc thi.]}
\end{problem}


\begin{problem}{\textbf{\textsc{A Tired Flappy Bird}}} A \href{https://flappybird.io/}{flappy bird} can jump multiple times in the air. Each time it jumps mid-air, it can suddenly change its speed and direction. For every jump, the bird can decide when to jump and in which direction. Between jumps, the bird falls freely under gravity, which pulls it down at the acceleration $g$. Say, our tired flappy bird starts off the cliff of height $H$ with the jumping velocity $V[1]=V_0$. Subsequent jumps in mid-air have decreasing velocities, i.e. the $n$-th jump has speed $V[n]=V_0/n$ ($n>1$). This majestic Vietnamese animal wants to travel as far as possible horizontally before it lands on the ground. Find the maximum horizontal distance the bird can travel (denoted as $L$ in the figure below) in meters, given that $H=100$m and $V_0=10$m/s. \textit{Note that each jumping velocity is the total speed of the bird after the jump (rather than e.g. adding to its speed before the jump).}

\FloatBarrier
\begin{figure*}[!htbp]
\centering
\includegraphics[width=0.5\textwidth]{problems/figures/flappybird.png}
\end{figure*}
\FloatBarrier

\end{problem}
\begin{problem}{\textbf{\textsc{A Tired Flappy Bird}}} A \href{https://flappybird.io/}{flappy bird} can jump multiple times in the air. Each time it jumps mid-air, it can suddenly change its speed and direction. For every jump, the bird can decide when to jump and in which direction. Between jumps, the bird falls freely under gravity, which pulls it down at the acceleration $g$. Say, our tired flappy bird starts off the cliff of height $H$ with the jumping velocity $V[1]=V_0$. Subsequent jumps in mid-air have decreasing velocities, i.e. the $n$-th jump has speed $V[n]=V_0/n$ ($n>1$). This majestic Vietnamese animal wants to travel as far as possible horizontally before it lands on the ground. Find the maximum horizontal distance the bird can travel (denoted as $L$ in the figure below) in meters, given that $H=100$m and $V_0=10$m/s. \textit{Note that each jumping velocity is the total speed of the bird after the jump (rather than e.g. adding to its speed before the jump).}

\FloatBarrier
\begin{figure*}[!htbp]
\centering
\includegraphics[width=0.5\textwidth]{problems/figures/flappybird.png}
\end{figure*}
\FloatBarrier

\end{problem}

\end{document}

