%\begin{problem}
%{\textbf{\textsc{An Envelope of Light}}} 
%A point light source on the ceiling is located at the center of a cylindrical housing (with an open base) of radius $R$ and height $H$. A wall is a horizontal distance $D$ from the center of the cylinder. Now consider a coordinate system with the light source at the origin. The wall, at $x=-D$, has the following shape in the $y-z$ plane:

%\begin{center}
    %\includegraphics[height=0.3\textwidth]{problems/figures/lightConeGraph.png}
    %\hspace{2em}
    %\includegraphics[height=0.3\textwidth]{problems/figures/lightConeHyperbola.png}
%\end{center}

%The vertical coordinate of the highest point of the curve observed is $-1.5\;\mathrm{m}$, while the gradients of the lines asymptotically tangent to the curve are $\pm 4/3$. On the right is shown an example setup of this phenomenon. Find the horizontal distance $D$ of the wall to the light source.
%\end{problem}
\begin{problem}
	{\textbf{\textsc{Bọc sáng}}} 
	Một nguồn sáng điểm trên trần được đặt tại trung tâm của một hộp hình trụ (có đáy mở) với bán kính $R$ với chiều cao $H$. Một bức tường nằm cách xa $D$ theo phương ngang từ trung tâm của hình trụ. Xét một hệ tọa độ với nguồn sáng tại gốc tọa độ. Bức tường, tại $x=-D$, có hình dạng sau trong mặt phẳng $y-z$:
	
	\begin{center}
		\includegraphics[height=0.3\textwidth]{problems/figures/lightConeGraph.png}
		\hspace{2em}
		\includegraphics[height=0.3\textwidth]{problems/figures/lightConeHyperbola.png}
	\end{center}
	
	Tọa độ dọc của điểm cao nhất trên đường cong quan sát được là $-1.5\;\mathrm{m}$, trong khi các hệ số góc của các đường thẳng tiếp tuyến tiệm cận với đường cong là $\pm 4/3$. Ở phía bên phải là một ví dụ về cách thiết lập hiện tượng này. Tìm khoảng cách ngang $D$ của bức tường từ nguồn sáng.
\end{problem}
