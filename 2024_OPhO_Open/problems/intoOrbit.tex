\begin{problem}{\textbf{\textsc{Into Orbit}}} Một khẩu pháo được cố định trên đỉnh của một mặt phẳng có chiều cao $R = 6.00 \times 10^6\;\mathrm{m}$, mặt phẳng này nằm trên một hành tinh có khối lượng $M = 6.00 \times 10^{24}\;\mathrm{kg}$ và bán kính $R$. Cả khẩu pháo và nền tảng mà nó đặt trên đều có khối lượng không đáng kể. Buồng pháo của khẩu pháo được nghiêng ngang so với hành tinh bên dưới. Khẩu pháo sau đó bắn một quả đạn có khối lượng $m = 45\;\mathrm{kg}$ qua một buồng có chiều dài $l=3\;\mathrm{m}$ với một gia tốc không đổi sao cho quả đạn có thể thành công vào một quỹ đạo elip xung quanh hành tinh. Lực tối thiểu mà khẩu pháo phải áp dụng lên quả đạn để làm điều này là bao nhiêu? Bạn có thể giả định rằng cả hành tinh và mặt phẳng đều không di chuyển trong quá trình này.\end{problem}