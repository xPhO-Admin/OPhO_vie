\begin{problem}

{\textbf{\textsc{Jaywalking}}} 
Bạn đang đi bộ băng qua đường tại một khoảng cách \( x = 8.00 \, \text{m} \) từ một ngã tư. Tại thời điểm \( t = 0 \), bạn bắt đầu đi qua ngã tư với một vận tốc không đổi vuông góc với hướng của đường, bạn cần thời gian \( T \) để hoàn tất việc băng qua. Cùng lúc đó, ở \( t = 0 \), có một chiếc ô tô tự lái (Tesla) cách ngã tư \( X = 40.0 \, \text{m} \) về phía bạn. Chiếc xe đi đến ngã tư với vận tốc sao cho nếu đèn tín hiệu vẫn xanh, nó sẽ đến ngã tư vào lúc \( t = T \).

Vào thời điểm \( t = 0 \), đèn tín hiệu chuyển sang màu đỏ, và do chiếc xe được điều khiển bằng máy tính, nó ngay lập tức bắt đầu giảm tốc với gia tốc không đổi \( a \) sao cho nó dừng lại chính xác tại \( X = 0 \).

Bạn có bị xe đụng phải không? Nếu có, hãy tính tốc độ của xe so với mặt đất khi nó đụng bạn, tính theo phần trăm của tốc độ ban đầu trước khi đèn đỏ chuyển màu. Nếu không, hãy tính khoảng cách của xe so với bạn khi bạn hoàn thành việc băng qua đường.

\end{problem}