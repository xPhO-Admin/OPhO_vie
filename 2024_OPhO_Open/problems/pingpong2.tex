%\begin{problem}{\textbf{\textsc{Ping-pong 2}}} We consider a serve with $n$ bounces before going over the net. The Olympic player is so incredibly good that he can control the direction of the velocity after each bounce as he pleases. Naturally more bounces decreases the minimal serving speed $v_n$. However, for some $N$, when $n\geq N$ the minimal serving speed no longer decreases if bounces are added, i.e. $N$ is the smallest natural number such that $v_m=v_N$ for all $m\geq N.$ Find $v_{N-1}^N$.

%\end{problem}
\begin{problem}{\textbf{\textsc{Bóng bàn 2}}} Ta xét một cú giao bóng với $n$ lần nảy trước khi đi qua lưới. Vận động viên Olympic giỏi đến mức anh ấy có thể điều khiển hướng vận tốc sau mỗi lần nảy theo ý muốn. Tất nhiên, nhiều lần nảy hơn sẽ làm giảm tốc độ giao bóng tối thiểu $v_n$. Tuy nhiên, với một số $N$, khi $n\geq N$ tốc độ giao bóng tối thiểu không còn giảm nữa khi thêm lần nảy, nghĩa là $N$ là số tự nhiên nhỏ nhất sao cho $v_m=v_N$ với mọi $m\geq N.$ Tìm $v_{N-1}^N$.
	
\end{problem}
