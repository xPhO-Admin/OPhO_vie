\begin{problem}
{\textbf{\textsc{Under The Lamplight}}} A large vat of the magical liquid Ophonium lies before you, with a depth of 5 meters. This liquid has a special property - its index of refraction changes variable to its depth! Its index of refraction can be expressed by the equation $$n = 1 + 2y$$ where $y$ is the depth of the liquid, in meters. A lamp, acting as a point source of light, is hung 3 meters above the vat. Light emanates from the lamp, casting its glow onto a circular section of the Ophonium's surface directly beneath it, covering an area of $3\pi$ square meters. As this light continues to travel downward, it enters the Ophonium, gradually penetrating its depths until it reaches the bottom of the vat. What is the area, in square meters, of the circle illuminated at the bottom of the vat? You may find the following integral useful:
$$\int\frac{1}{\sqrt{x^2-1}}dx = \cosh^{-1}(x) + C$$
\end{problem}