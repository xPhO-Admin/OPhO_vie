% \begin{solution}
% The key idea is to realize that the 1D probability density function is analogous to the intensity from single slit diffraction. More precisely, let us consider the 1D probability density at a given position $x$ in the plane of the detector. $P = \norm{\Psi_x}^2 = \Psi_x ^{\ast} \Psi_x$ corresponds to taking the square of the norm of summed phasors, which is analogous to optical intensity being the square of the norm of the electric field phasor.
% \newline
% \newline
% This makes our life easier, because we can simply reuse the derivations from classical wave optics! The DeBroglie wavelength of the neutrons is $$\lambda = \frac{h}{p} = \frac{h}{mv} = 601.1\;\mathrm{nm}$$ 

% As can be derived using phasors (or simply by reusing the analogous formula from wave optics), we have
% $$\norm{\Psi_x}^2 \propto {\left[\frac {\sin \left(\frac {\pi d \frac{x}{\sqrt{r^2 + x^2}}}{\lambda}\right)}{\left(\frac {\pi d \frac{x}{\sqrt{r^2 + x^2}}}{\lambda}\right)}\right]^2} $$
% We have to be careful to normalize the probability density function when calculating the final result:
% $$P = \frac {\int_{-w/2}^{w/2}\norm{\Psi_x}^2\,dx}{\int_{-\infty}^{\infty}\norm{\Psi_x}^2\,dx} $$
% Plugging in the given parameters, this gives us a final percentage of $$P\times100\% = \boxed{90.2\%}$$


% \end{solution}

\begin{solution}
Ý tưởng chính là nhận ra rằng hàm mật độ xác suất 1D tương tự như cường độ từ nhiễu xạ khe đơn. Cụ thể hơn, hãy xem xét hàm mật độ xác suất 1D tại một vị trí $x$ trong mặt phẳng của máy dò. $P = \norm{\Psi_x}^2 = \Psi_x ^{\ast} \Psi_x$ tương ứng với việc lấy bình phương của chuẩn của các pha, điều này tương tự như cường độ quang học là bình phương của chuẩn của pha trường điện.
\newline
\newline
Điều này làm cho cuộc sống của chúng ta dễ dàng hơn, vì chúng ta có thể đơn giản sử dụng lại các phép tính từ quang học sóng cổ điển! Bước sóng DeBroglie của neutron là $$\lambda = \frac{h}{p} = \frac{h}{mv} = 601.1\;\mathrm{nm}$$ 

Như có thể được suy ra bằng cách sử dụng các pha (hoặc đơn giản bằng cách sử dụng lại công thức tương tự từ quang học sóng), chúng ta có
$$\norm{\Psi_x}^2 \propto {\left[\frac {\sin \left(\frac {\pi d \frac{x}{\sqrt{r^2 + x^2}}}{\lambda}\right)}{\left(\frac {\pi d \frac{x}{\sqrt{r^2 + x^2}}}{\lambda}\right)}\right]^2} $$
Chúng ta phải cẩn thận để chuẩn hóa hàm mật độ xác suất khi tính toán kết quả cuối cùng:
$$P = \frac {\int_{-w/2}^{w/2}\norm{\Psi_x}^2\,dx}{\int_{-\infty}^{\infty}\norm{\Psi_x}^2\,dx} $$
Thay các tham số đã cho vào, điều này cho chúng ta phần trăm cuối cùng là $$P\times100\% = \boxed{90.2\%}$$


\end{solution}
