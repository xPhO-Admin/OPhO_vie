\begin{solution} Ý tưởng chính là viết ánh sáng phân cực phẳng như sự chồng chất của hai sóng ánh sáng phân cực tròn có cùng tần số.Ý tưởng chính là viết ánh sáng phân cực phẳng như sự chồng chất của hai sóng ánh sáng phân cực tròn có cùng tần số.
\begin{center}
    \includegraphics[width=0.25\textwidth]{solutions/figures/polarization-decomposition.png}
\end{center}
    
Sau khi đi qua dung dịch, do chúng có chỉ số khúc xạ khác nhau là $n_L$ và $n_R$, độ chênh lệch pha giữa chúng được cho bởi
$$\frac{\phi}{2\pi} = \frac{L}{\lambda_{\text{medium}}} = \frac{L}{\frac{\lambda}{n}} \implies \Delta \phi = \frac{2\pi}{\lambda}L\Delta n$$
    
Tuy nhiên, cần lưu ý rằng đây là độ dịch pha chứ không phải góc quay.

\begin{center}
    \includegraphics[width=0.3\textwidth]{solutions/figures/rotation-shift.png}
\end{center}
    
Do đó $\Delta \theta = \frac{1}{2}\Delta \phi = \frac{\pi}{\lambda} L \Delta n$. Thay các giá trị vào ta được $\boxed{\Delta \theta = \frac{3\pi}{2} \approx 4.71\;\mathrm{rad}}$.

\end{solution}