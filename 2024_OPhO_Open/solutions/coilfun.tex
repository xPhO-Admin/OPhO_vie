\begin{solution}
The first key idea here is that the magnetic flux through the solenoid $\Phi = LI$ is a conserved quantity, since $\varepsilon = \frac{d \Phi}{dt}$ is necessarily zero to stop current from blowing up. The second key idea is that this is essentially just a conservation-of-energy question, with the decrease in magnetic field energy during the transit of the core through the solenoid being converted to kinetic energy. As a result, the optimal time to shut down the solenoid occurs when magnetic field energy is lowest i.e. when the core is completely inside the solenoid.
\newline
\newline
Since B is conserved and $\mu \gg \mu_0$, there is essentially no magnetic field energy when the solenoid should be shut down.
As a result, we have
$$E_B = \frac{\bigg (\frac{\mu_0 N I}{L}\bigg )^2}{2 \mu_0}Al = {\frac{1}{2}}mv^2$$
Plugging in yields $\boxed{v = 179.4 \;\mathrm{m/s}}$
\end{solution}