\begin{solution}
Ý tưởng chính đầu tiên ở đây là từ thông qua ống dây \( \Phi = LI \) là một đại lượng được bảo toàn, vì \( \varepsilon = \frac{d \Phi}{dt} \) nhất thiết phải bằng không để ngăn dòng điện tăng vọt. Ý tưởng chính thứ hai là đây về cơ bản chỉ là một bài toán bảo toàn năng lượng, với việc giảm năng lượng từ trường trong quá trình lõi di chuyển qua ống dây được chuyển thành động năng. Do đó, thời điểm tối ưu để tắt ống dây xảy ra khi năng lượng từ trường là thấp nhất, tức là khi lõi hoàn toàn ở trong ống dây.
\newline
\newline
Vì B ược bảo toàn và  $\mu \gg \mu_0$, về cơ bản không còn năng lượng từ trường khi ống dây nên được tắt.
Từ đó, ta có
$$E_B = \frac{\bigg (\frac{\mu_0 N I}{L}\bigg )^2}{2 \mu_0}Al = {\frac{1}{2}}mv^2$$
Thay vào sẽ cho ra $\boxed{v = 179.4 \;\mathrm{m/s}}$
\end{solution}