%%Thanks to Joshua Wang (USPT '24) for providing the non-fakesolve solution! :)
\begin{solution}
Let the string have tension $T$, and let the displacement be $A\sin(\pi x/L)\sin(\omega t)$. We compute the total energy by finding the length at the peak of an oscillation:
\begin{align*}E&=T(L_{\text{peak}} - L)\\ &=T\left(\int_0^L\sqrt{1 + \frac{\pi^2 A^2}{L^2}\cos^2\frac{\pi x}{L}}\ dx - L\right)\\ &\approx T\left(\int_0^L 1 + \frac{\pi^2 A^2}{2L^2}\cos^2\frac{\pi x}{L}\ dx - L\right)\\ &= \frac{\pi^2 A^2 T}{4L}\end{align*}
At time $t$, the net horizontal force on the finger is:
\begin{align*}F_x = T\left(1 - \cos\left(\arctan\left(\frac{A\pi}{L}\sin(\omega t)\right)\right)\right)\approx \frac{\pi^2 A^2 T}{2L^2}\sin^2(\omega t)\end{align*}
Thus, the average horizontal force on the finger is $\displaystyle \langle F_x\rangle = -\frac{dE}{dL} = \frac{\pi^2 A^2 T}{4L^2} = \frac{E}{L}$. This gives $\displaystyle E \propto \frac{1}{L}$, so $A$ is constant, and $A_2=A=\boxed{1.25\times 10^{-3}\;\mathrm{m}}.$
\newline
\newline
Note that the question could also have been solved through the adiabatic invariant, or alternately by realizing that the restoring force is linear in displacement at all points on the string. As a result, a valid solution, by superposition, is to use a rotational analogy (with the string performing rotation like a skipping-rope) and conserve angular momentum. In that case, it will be observed that the decrease in mass of the vibrating portion of the string exactly counteracts the increase in angular velocity caused by the decreased length.  

\end{solution}