\begin{solution}

Here we can use the fact that for a conducting material, the field near 
the surface is perpendicular to the conductor. First we'll get the charge 
distribution of a disk from \eqref{ellipsoidCharge}, then make an analogy 
between electrostatics and gravity.

Let $x^2+y^2 = r^2$, $a=b=R$ and $c\rightarrow 0$.
\begin{gather*}
    \frac{x^2}{a^2} + \frac{y^2}{b^2} + \frac{z^2}{c^2} = 1\\
    \frac{r^2}{R^2}+\frac{z^2}{c^2}=1\\
    \frac{z^2}{c^2} = 1 - \frac{r^2}{R^2} 
\end{gather*}
Now $\sigma$ becomes:
\begin{equation*}
    \sigma = \frac{q}{4\pi R^2c} {\left( \frac{r^2}{R^4}
    + \frac{1 - \frac{r^2}{R^2}}{c^2} \right)} ^ {-1 / 2}
\end{equation*}

But $c^{-2} \gg a^{-2}$ so:
\begin{gather*}
    \sigma \approx \frac{q}{4\pi R^2c} {\left(\frac{1 - \frac{r^2}{R^2}}{c^2}
    \right)} ^ {-1 / 2} 
\end{gather*}

And this gives the charge distribution on a charged disk:
\begin{equation*}
    \sigma = \frac{q}{4\pi R^2} {\left(1 - \frac{r^2}{R^2}
    \right)} ^ {-1 / 2} 
\end{equation*}

Actually, this is the charge distribution on one of the faces of the disk.
Because we let $c \rightarrow 0$ we have to multiply this by two(you can 
check that the integral would give half the charge for the formula above).
\begin{equation*}
    \sigma = \frac{q}{2\pi R^2} {\left(1 - \frac{r^2}{R^2}
    \right)} ^ {-1 / 2} 
\end{equation*}

The electric field given by a surface charge $\sigma$ near the surface is 
\[ \mathrm{E} = \frac{\sigma}{2\varepsilon_0} \]
And the gravitational field near the surface(also Gauss's law):
\[ 2 \mathrm{\Gamma} \mathrm{dS}= 4 \pi \mathrm{G} \rho h \mathrm{dS}\]
\[ \mathrm{\Gamma} = 2 \pi \mathrm{G} \rho h\]
So $\rho h$ acts similarly to $\sigma$. ($\Gamma$ is the gravitational 
acceleration)

\begin{equation}
    \rho = \rho_0 {\left(1 - \frac{r^2}{R^2} \right)} ^ {-1 / 2} 
    \label{densityDistribution}
\end{equation}

Now for the numerical part,
% (1 - 1 / 9) ^ ( - 1 / 2) / (1 - 4 / 9) ^ ( - 1 / 2)
\begin{equation*}
    \frac{\rho_{\frac{R}{3}}}{\rho_{\frac{2R}{3}}} =
    \sqrt{\frac{1 - \frac{4}{9}}{1 - \frac{1}{9}}} = 0.7905694
\end{equation*}

\end{solution}