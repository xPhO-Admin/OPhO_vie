\begin{solution}

Tốc độ ban đầu tối thiểu cần thiết để vào quỹ đạo sẽ xảy ra khi bán trục chính của quỹ đạo được tối thiểu hóa. Trục bán chính này đạt giá trị nhỏ nhất khi nó bằng $1.5R$.

Theo phương trình vis-viva, vận tốc tối thiểu cần thiết để vào quỹ đạo từ vị trí phóng của pháo là

$$v = \sqrt{GM\bigg(\frac{2}{2R}-\frac{1}{a}\bigg)} = \sqrt{GM\bigg(\frac{1}{R}-\frac{1}{1.5R}\bigg)} = \sqrt{\frac{GM}{3R}}$$

Quả pháo cần gia tốc để đạt đến vận tốc này khi nó chạm đến cuối nòng pháo. Gia tốc có thể tính như sau:

$$v_f^2 = v_i^2 + 2a\Delta x$$

$$\bigg(\sqrt{\frac{GM}{3R}}\bigg)^2 = 2al\longrightarrow a = \frac{GM}{6Rl}$$

Do đó, lực tối thiểu là:

$$ F = ma = \frac{GMm}{6Rl}\approx \boxed{1.67\times10^8\;\mathrm{N}}$$

\end{solution}