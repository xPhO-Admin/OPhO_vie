\begin{solution}
    Start by looking at a cross section of the liquid through the central axis. Consider a small mass $dm$ of liquid at the surface, a distance $x$ from the axis of rotation. The forces can be balanced as follows:

$$dN\cos{\theta} = gdm$$

$$dN\sin{\theta} = x\omega^2dm$$

$$\tan{\theta} = \frac{dy}{dx} = \frac{x\omega^2}{g}$$

$$\int dy = \int\frac{x\omega^2}{g}dx\longrightarrow y = \frac{x^2\omega^2}{2g}$$

This equation is parabolic, so the liquid forms a paraboloid lens. The focal point can then be calculated as 

$$f = \bigg(0, \frac{g}{2\omega^2}\bigg)$$

Because light is incident vertically, photons will either collide with the lens once or twice.

At the brim of the cylindrical well, the height of the liquid relative to its height at the center of the well is

$$\frac{\omega^2}{2g}$$

The points at which incident photons will collide with the lens twice can now be calculated.

$$y = -\bigg(\frac{\omega^2}{2g} - \frac{g}{2\omega^2}\bigg)x + \frac{g}{2\omega^2}$$

$$y = \frac{x^2\omega^2}{2g}$$

$$\frac{g}{2\omega^2} - \frac{x^2\omega^2}{2g} = \bigg(\frac{\omega^2}{2g} - \frac{g}{2\omega^2}\bigg)x$$

$$\frac{x^2\omega^2}{2g} + \bigg(\frac{\omega^2}{2g} - \frac{g}{2\omega^2}\bigg)x - \frac{g}{2\omega^2} = 0$$

$$x = -1, \frac{g^2}{\omega^4}$$

Any photon initially incident at least $\frac{g^2}{\omega^4}$ away from the central axis will thus hit the mirror twice. The rest of the photons will only hit the mirror once. The value of $n$ can now be calculated.

$$n = 2 - \frac{g^4}{\omega^8} = 2 - \frac{9.8^4}{5^8} \approx \boxed{1.976}$$


\end{solution}