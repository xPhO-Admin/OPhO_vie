\begin{solution}
    Bắt đầu bằng cách xem xét một mặt cắt của chất lỏng qua trục trung tâm. Xét một phần khối lượng nhỏ $dm$ của chất lỏng ở bề mặt, cách trục xoay một khoảng $x$ tính từ trục quay. Các lực cân bằng như sau:

$$dN\cos{\theta} = gdm$$

$$dN\sin{\theta} = x\omega^2dm$$

$$\tan{\theta} = \frac{dy}{dx} = \frac{x\omega^2}{g}$$

$$\int dy = \int\frac{x\omega^2}{g}dx\longrightarrow y = \frac{x^2\omega^2}{2g}$$

Phương trình này là parabol, do đó, chất lỏng tạo thành thấu kính dạng paraboloid. Tiêu điểm có thể được tính như sau:

$$f = \bigg(0, \frac{g}{2\omega^2}\bigg)$$

Vì ánh sáng tới theo phương thẳng đứng, photon sẽ va chạm với thấu kính một hoặc hai lần.

Tại mép của giếng trụ, độ cao của chất lỏng so với độ cao tại trung tâm giếng là

$$\frac{\omega^2}{2g}$$

Các điểm tại đó photon tới sẽ va chạm với thấu kính hai lần có thể được tính toán như sau:

$$y = -\bigg(\frac{\omega^2}{2g} - \frac{g}{2\omega^2}\bigg)x + \frac{g}{2\omega^2}$$

$$y = \frac{x^2\omega^2}{2g}$$

$$\frac{g}{2\omega^2} - \frac{x^2\omega^2}{2g} = \bigg(\frac{\omega^2}{2g} - \frac{g}{2\omega^2}\bigg)x$$

$$\frac{x^2\omega^2}{2g} + \bigg(\frac{\omega^2}{2g} - \frac{g}{2\omega^2}\bigg)x - \frac{g}{2\omega^2} = 0$$

$$x = -1, \frac{g^2}{\omega^4}$$

Bất kỳ photon nào ban đầu cách trục trung tâm ít nhất là $\frac{g^2}{\omega^4}$ sẽ va chạm với gương hai lần. Các photon còn lại chỉ va chạm với gương một lần. Giá trị của $n$ có thể được tính như sau.

$$n = 2 - \frac{g^4}{\omega^8} = 2 - \frac{9.8^4}{5^8} \approx \boxed{1.976}$$


\end{solution}