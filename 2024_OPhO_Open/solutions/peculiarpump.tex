% \begin{solution}
% Let $N$ be the number of particles in the box as a function of time, and let $T_1$ be the temperature of the box as a function of time. Then effusion gives

% \begin{equation}
%     \frac{dN}{dt}=\frac{2AN}{V}\sqrt{\frac{k_BT_1}{2\pi m}}
% \end{equation}

% Typically there would be no factor of $2$ and $\frac{dN}{dt}$ would be negative, but $3$ new particles enter the box from the reservoir every time one leaves from the hole. \\
% \\
% We also know that energy leaves the box through the hole through the thermal energy of the leaving gas. A similar process happens to the gas entering from the reservoir. It can be shown that the average energy of a particle effusing out of a small hole is $2kT$, and since the particles coming in from the reservoir are also effusing in, this is the average energy of the incoming and outgoing particles. The rate of gas leaving is $\frac{1}{2}\frac{dN}{dt}$ and the rate of gas entering is $\frac{3}{2}\frac{dN}{dt}$, so

% \begin{align}
%     \frac{d}{dt}\left(\frac{3}{2}Nk_BT_1\right)&=\left(\frac{3}{2}\frac{dN}{dt}\right)(2k_BT_2)-\left(\frac{1}{2}\frac{dN}{dt}\right)(2k_BT_1)
% \end{align}
%     Simplifying yields 

%     \begin{equation}
%         \frac{dT_1}{dt}=\frac{1}{N}\frac{dN}{dt}\left(2T_2-
%         \frac{5}{3}T_1\right)
%     \end{equation}
%     Plugging in for dN/dt yields

%     \begin{equation}
%         \frac{dT_1}{dt}=\frac{A}{V}\sqrt{\frac{k_BT_1}{2\pi m}}\left(4T_2-\frac{10}{3}T_1\right)
%     \end{equation}
%     Integrating this yields
%     \begin{equation}
%         t=(0.007524)\frac{V}{A}\sqrt{\frac{2\pi m}{k_B}}
%     \end{equation}
%     Which, plugging in the numbers, is 0.492 days. 
    
% \end{solution}

\begin{solution}
Gọi $N$ là số lượng hạt trong hộp theo thời gian, và gọi $T_1$ là nhiệt độ của hộp theo thời gian. Sau đó, sự thoát khí cho ta

\begin{equation}
    \frac{dN}{dt}=\frac{2AN}{V}\sqrt{\frac{k_BT_1}{2\pi m}}
\end{equation}

Thông thường sẽ không có hệ số $2$ và $\frac{dN}{dt}$ sẽ âm, nhưng $3$ hạt mới vào hộp từ bể chứa mỗi khi một hạt rời khỏi lỗ. \\
\\
Chúng ta cũng biết rằng năng lượng rời khỏi hộp qua lỗ thông qua năng lượng nhiệt của khí thoát ra. Một quá trình tương tự xảy ra với khí vào từ bể chứa. Có thể chứng minh rằng năng lượng trung bình của một hạt thoát ra khỏi một lỗ nhỏ là $2kT$, và vì các hạt vào từ bể chứa cũng thoát ra, đây là năng lượng trung bình của các hạt vào và ra. Tốc độ khí rời khỏi là $\frac{1}{2}\frac{dN}{dt}$ và tốc độ khí vào là $\frac{3}{2}\frac{dN}{dt}$, do đó

\begin{align}
    \frac{d}{dt}\left(\frac{3}{2}Nk_BT_1\right)&=\left(\frac{3}{2}\frac{dN}{dt}\right)(2k_BT_2)-\left(\frac{1}{2}\frac{dN}{dt}\right)(2k_BT_1)
\end{align}
    Đơn giản hóa cho ta 

    \begin{equation}
        \frac{dT_1}{dt}=\frac{1}{N}\frac{dN}{dt}\left(2T_2-
        \frac{5}{3}T_1\right)
    \end{equation}
    Thay $\frac{dN}{dt}$ vào ta có

    \begin{equation}
        \frac{dT_1}{dt}=\frac{A}{V}\sqrt{\frac{k_BT_1}{2\pi m}}\left(4T_2-\frac{10}{3}T_1\right)
    \end{equation}
    Tích phân điều này cho ta
    \begin{equation}
        t=(0.007524)\frac{V}{A}\sqrt{\frac{2\pi m}{k_B}}
    \end{equation}
    Thay các số vào, ta có 0.492 ngày. 
    
\end{solution}
